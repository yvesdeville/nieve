%% \chapter{GEV}

%% ===========================================================================
%% Copyright Yves Deville <deville.yves@alpestat.com> 2022
%%
%% Mind that this file required a serious amount of work and should not be
%% copied without citing the source.
%% ===========================================================================
\newpage

\section*{Note}
In this report

\begin{itemize}
\item The expressions computed by \textbf{Maxima} are reported
  {\color{MonVertF}in green}.
  
\item The expressions involving some manual calculations are reported
  {\color{red}in red}.  These are intended to be easier to implement
  as they make use of ``auxiliary quantities'' that may be reused in
  several expressions.
\end{itemize}

The simplifications are not needed in the section devoted to the
quantile but they help much for the log-density and the distribution.
Although they have been carfully checked, the expressions
{\color{red}in red} could still contain some errors.  Please report
any error at \url{ https://github.com/yvesdeville/nieve/issues/}.

\section{Auxiliary variables}

\begin{maxima}
  V: 1 + xi * z
\end{maxima}
\begin{maxima}
  W: V^(-1/xi)
\end{maxima}

In order to evaluate the the log-density $\log f(y)$ or the
distribution function $F(y)$ for some $y$,  the following auxiliary variables
will be used
\begin{equation}
  \label{eq:AuxVar}
  \left\{ \rule{0.5em}{0em}
  \begin{aligned}
    z &:= \frac{y - \mu}{\sigma}, \\
    V &:= 1 + \xi z, \\
    W &:= V^{-1/\xi},\\
    T &:= \frac{\log V}{\xi^2} - \frac{z}{\xi V},\\
    R &:= \frac{1}{\xi} \left[2T - \frac{z^2}{V^2} \right].
  \end{aligned}
  \right.
\end{equation}
In practice $z$ should take quite moderate value, say between $-30$
and $30$ and if this is not the case, the wanted results can be
anticipated.  It is worth noting that for fixed values of $y$, of the
location $\mu$ and the scale $\sigma$ these quantities all have a
finite limit and Taylor expansion for $\xi \approx 0$
\begin{equation*}
  \begin{aligned}
    W
    &= e^{-z} + \frac{z^2e^{-z}}{2} \, \xi - \frac{z^3e^{-z} [3z -8]}{24}  \, \xi^2  + o(\xi^2),\\
    T
    &= \frac{z^2}{2} - \frac{2z^3}{3} \xi + \frac{3z^4}{4}\xi^2 + o(\xi^2),\\
    R
    &= -\frac{z^3}{3} + \frac{z^4}{2} \, \xi - \frac{3z^5}{5}\,\xi^2 + o(\xi^2). 
  \end{aligned}
\end{equation*}
So although their are not defined when $\xi = 0$, the quantities $W$,
$T$ and $R$ can be prolongated by continuity at $\xi = 0$. The
log-density, the distribution functions and their derivatives can be
expressed with this auxiliary variables. Although $W$ and $T$ require
the use of trancendental functions: non-fractional power and log, the
functions are subsequently obtained multiplications and divisions of
these quantities. So if the auxiliary variables of~(\ref{eq:AuxVar})
are safely evaluated in the wanted range of values for $y$ and the GEV
parameters, then all the results of interest will be evaluated safely
and efficiently.


\section{Log-density $\log f$}

\subsection{Expression}

\begin{maxima}
  U: (1 + \xi - V^(-1/xi)) / sigma / V
\end{maxima}

%%-----------------------------------------------------------------------
\begin{maxima}
  g: -log(sigma) - (1 / xi + 1) * log(V) - V^(-1 / xi)
\end{maxima}
%
\begin{maxima}
  gz: diff(g, z, 1)
\end{maxima}%
%% 
\begin{maxima}
  gxi: diff(g, xi, 1)
\end{maxima}%
%%
% -----------------------------------------------------------------------

The log-likelihood is defined by
\[
\log f = 
\begin{maxima}
  tex(g)
\end{maxima}
\]

\begin{maxima}
  g: -log(sigma) - (1 / xi + 1) * log(V) - V^(-1 / xi)
\end{maxima}

{\color{red}
  \begin{equation}
    \log f  := - \log \sigma - (\xi + 1) \left[\xi T + \frac{z}{V}\right] - W
  \end{equation}
}

% \subsection{Derivatives}
% %%=======================================================

% \begin{align*}
%   \frac{\partial \log f}{\partial z}
%   &=
%     \begin{maxima}
%       tex(gz)
%     \end{maxima}
%   \\
%   \frac{\partial \log f}{\partial \xi}
%   &=
%     \begin{maxima}
%       tex(gxi)
%     \end{maxima}
% \end{align*}


\subsection{Taylor expansion at $\xi = 0$}
%%=======================================================

\begin{maxima}
  txi: taylor(g, xi, 0, 2)
\end{maxima}
{\color{MonVertF}
\[
  \log f = 
  \begin{maxima}
    tex(txi)
  \end{maxima}
\]
}

which can be simplified as
\begin{equation}
  \label{eq:TylorLogf}
  \color{red}
  \log f = -\left[ \log \sigma + z + e^{-z} \right] + \frac{1}{2}\,
  \left[ (1 -  e^{-z}) z -2  \right]\,z \xi - \frac{1}{24} \,
  \left[ 3 z^2 e^{-z} + 8 z (1 - e^{-z}) - 12 \right] z^2 \xi^2
\end{equation}

\subsection{First-order derivatives: expression}
%%=======================================================

\subsubsection*{Remark: scaling}
%%-----------------------------------
In order to get quite simple expressions, we remark that $\log f$
depends on $\mu$ only through $z$, so
$$
\frac{\partial \log f}{\partial \mu} = \frac{\partial \log f}{\partial z} \,
   \dfrac{\partial z}{\partial \mu} = -\frac{1}{\sigma} \, \frac{\partial \log f}{\partial z}
$$ 
and up to the term $-\log \sigma$ in $\log f$, the same is true for
$\sigma$ so it is simpler to compute the derivative w.r.t. $z$ and then find the
derivatives w.r.t. $\mu$ and $\sigma$ using
{\color{red}
\begin{equation}
  \begin{aligned}
    \frac{\partial \log f}{\partial \mu}
    &= -\frac{1}{\sigma} \, \frac{\partial \log f}{\partial z}\\
    \frac{\partial \log f}{\partial \sigma}
    %% = -\frac{1}{\sigma} + \frac{\partial \log f}{\partial z} \, \dfrac{\partial z}{\partial \sigma} =
    &=-\frac{1}{\sigma} -\frac{z}{\sigma} \, \frac{\partial \log f}{\partial z}.
  \end{aligned}
\end{equation}
}

\subsubsection*{Raw expressions}
%%-----------------------------------
Here are the raw expressions found by Maxima
\begin{equation*}
\color{MonVertF}
  \begin{aligned}
  \frac{\partial \log f}{\partial z}
  &=
    \begin{maxima}
      tex(gz)
    \end{maxima}
    \\
    \frac{\partial \log f}{\partial \xi}
  &=
    \begin{maxima}
      tex(gxi)
    \end{maxima}
  \end{aligned}
\end{equation*}

\subsubsection*{Simplified expressions}

We use the following simplifications
{\color{red}
  \begin{equation}
    \begin{aligned} 
      \frac{\partial \log f}{\partial z}
      %% &= - \sigma U,
      &= \frac{1}{V} \, \left[W - \xi - 1 \right]   
      \\
      \frac{\partial \log f}{\partial \xi}
      %% &= \frac{1}{\xi^2} \, [1 - V^{-1/\xi}]  \, \log V - z \, U \frac{\sigma}{\xi}.
      &= T \left[1 - W \right] - \frac{z}{V}
    \end{aligned}
  \end{equation}
}


\subsection{First-order derivatives: Taylor expansions for $\xi \approx 0$}
\begin{maxima}
  tU: taylor(U, xi, 0, 1)
\end{maxima}
\begin{maxima}
  tgz: taylor(gz, xi, 0, 1)
\end{maxima}
\begin{maxima}
  tgxi: taylor(gxi, xi, 0, 1)
\end{maxima}

% {\color{MonVertF}
% $$
% U =
% \begin{maxima}
%    tex(tU)
% \end{maxima}
% $$
% }

\begin{equation*}
  \color{MonVertF}
  \begin{aligned}
    \frac{\partial \log f}{\partial z}
    &= 
      \begin{maxima}
        tex(tgz)
      \end{maxima} \\
    \frac{\partial \log f}{\partial \xi}
    &= 
      \begin{maxima}
        tex(tgxi)
      \end{maxima}
  \end{aligned}
\end{equation*}

\begin{equation}
  \color{red}
  \label{eq:TaylorDer1Logf}
  \begin{aligned}
    \frac{\partial \log f}{\partial z}
    &= -\left[ 1 - e^{-z} \right]
      - \frac{1}{2}\, \left[2 -2 (1 - e^{-z}) z -  e^{-z} z^2 \right] \, \xi
      + o(\xi)  \\
    \frac{\partial \log f}{\partial \xi}
    &= -\frac{1}{2}\,\left[ 2 z - (1 - e^{-z}) z^2 \right]
      +  \frac{1}{12}\, \left[12 - 8 (1 - e^{-z}) z - 3z^2\right] \, z^2 \xi
      + o(\xi) 
  \end{aligned}
\end{equation}


\subsubsection*{Limits for $\xi \to 0$}
%%========================================================
Here are the limits for $\xi \to 0$
{\color{MonVertF}
  \begin{align*}
  \lim_{\xi \to 0} \frac{\partial \log f}{\partial z}
  &=
    \begin{maxima}
      tex(limit (gz, xi, 0))
    \end{maxima}
  \\
  \lim_{\xi \to 0} \frac{\partial \log f}{\partial \xi}
  &=
    \begin{maxima}
      tex(limit (gxi, xi, 0))
    \end{maxima}
\end{align*}}
that is
\begin{equation}
  \label{eq:LimDer1Logf}
  \color{red}
  \begin{aligned}
    \lim_{\xi \to 0} \frac{\partial \log f}{\partial z}
    &= 1 - e^{-z} \\
    \lim_{\xi \to 0} \frac{\partial \log f}{\partial \xi}
    &= -\frac{z}{2} \,\left[2 - (1 - e^{-z}) z \right] 
  \end{aligned}
\end{equation}


\subsection{Second-order derivatives: expressions}
%% ============================================
\subsubsection*{Scaling}
%% -----------------------------------
{\color{red}
  \begin{equation}
    \begin{aligned}
      \frac{\partial^2 \log f}{\partial \mu^2}
      &= \frac{1}{\sigma^2} \,
        \frac{\partial^2 \log f}{\partial z^2} \\
      \frac{\partial^2 \log f}{\partial \mu \partial \sigma}
      &= \frac{1}{\sigma^2} \left\{
        \frac{\partial \log f}{\partial z} + z
        \frac{\partial^2 \log f}{\partial z^2}
        \right\}\\
      \frac{\partial^2 \log f}{\partial \mu \partial \xi}
      &= \frac{-1}{\sigma} \,
        \frac{\partial^2 \log f}{\partial z \partial \xi}\\
      \frac{\partial^2 \log f}{\partial \sigma^2}
      &= \frac{1}{\sigma^2} +
        \frac{z}{\sigma^2} \left\{
    2 \, \frac{\partial \log f}{\partial z}  +
        z \,
        \frac{\partial^2 \log f}{\partial z^2}
        \right\}\\
      \frac{\partial^2 \log f}{\partial \sigma \partial \xi}
      &= \frac{-z}{\sigma} \,
        \frac{\partial^2 \log f}{\partial z \partial \xi} 
    \end{aligned}
  \end{equation}
}

\subsubsection*{Raw expressions}
%%-----------------------------------
\begin{maxima}
  gzz: diff(g, z, 2)
\end{maxima}%

\begin{maxima}
  gzzb: letsimp(gzz)
\end{maxima}%

\begin{maxima}
  gzxi: diff(g, z, 1, xi, 1)
\end{maxima}%

\begin{maxima}
  gzxib: letsimp(gzxi)
\end{maxima}%

\begin{maxima}
  gxixi: diff(g, xi, 2)
\end{maxima}%

\begin{maxima}
  gxixib: letsimp(gxixi)
\end{maxima}%

% %% \begin{landscape}
% % %   -------------------------------------
{\color{MonVertF} \small
\begin{align*}
  \frac{\partial^2 \log f}{\partial z^2}
  &=
  \begin{maxima}
    tex(gzzb)
  \end{maxima}\\
  \frac{\partial^2 \log f}{\partial z \partial \xi}
  &=
  \begin{maxima}
    tex(gzxib)
  \end{maxima}\\
  \frac{\partial^2 \log f}{\partial \xi^2}
  &=
  \begin{maxima}
    tex(gxixib)
  \end{maxima}
\end{align*}
}

\subsubsection*{Simplified expressions}
%%-----------------------------------
Using the auxiliary variables~(\ref{eq:AuxVar})
\begin{equation}
  \color{red}
  \label{eq:SimDer2Logf}
  \begin{aligned}
    \frac{\partial^2 \log f}{\partial z^2}
    &= - \frac{\xi + 1}{V^2}\, \left[W - \xi\right] \\
    \frac{\partial^2 \log f}{\partial z \partial \xi}
    &=  \frac{W}{V}\, \left[T - \frac{z}{V} \right] - \frac{1}{V}
      + (\xi + 1) \frac{z}{V^2}\\
    \frac{\partial^2 \log f}{\partial \xi^2} 
    &= - W\left\{T^2 - R  \right\} - R +  \frac{z^2}{V^2}
  \end{aligned}
\end{equation}




% \section{Distribution function $F$}
\subsection{Second-order derivatives: limits for $\xi \to 0$}
%%========================================================
Here are the limits for $\xi \to 0$
{\color{MonVertF}
  \begin{align*}
  \lim_{\xi \to 0} \frac{\partial^2 \log f}{\partial^2 z^2}
  &=
    \begin{maxima}
      tex(limit (gzz, xi, 0))
    \end{maxima}
  \\
  \lim_{\xi \to 0} \frac{\partial^2 \log f}{\partial z \partial \xi}
  &=
    \begin{maxima}
      tex(limit (gzxi, xi, 0))
    \end{maxima}
  \\
  \lim_{\xi \to 0} \frac{\partial^2 \log f}{\partial^2 \xi^2}
   &=
    \begin{maxima}
      tex(limit (gxixi, xi, 0))
    \end{maxima}
\end{align*}
}

{\color{red}
\begin{align*}
  \lim_{\xi \to 0} \frac{\partial^2 \log f}{\partial^2 z^2}
  &= -e^{-z}
  \\
  \lim_{\xi \to 0} \frac{\partial^2 \log f}{\partial z \partial \xi}
  &= \frac{1}{2} \, \left[ -2  - 2 \,(1 - e^{-z}) \, z  + e^{-z} z^2\right]  
  \\
  \lim_{\xi \to 0} \frac{\partial^2 \log f}{\partial^2 \xi^2}
   &=
    \frac{z^2}{12} \, \left[12 + 8 \, (1 - e^{-z}) \, z -3 \,e^{-z} z^2 \right]
\end{align*}}

\section{Distribution function $F$}

\subsection{Expression}

\begin{maxima}
  W: V^(-1/xi)
\end{maxima}

Using the auxiliary variables~(\ref{eq:AuxVar})
% $$
% W := V^{-1/\xi} = [1 + \xi z]^{-1/\xi}, \qquad T := \frac{\log V}{\xi^2}
% - \frac{z}{\xi V}
% $$
% Then
\begin{equation}
  \color{red}
  \label{eq:DefF}
  F := \exp\{ - W \}.
\end{equation}

\begin{maxima}
  F: exp(-W)
\end{maxima}

\subsection{Taylor expansions for $\xi \approx 0$}
\subsubsection*{Raw expression}

\begin{maxima}
  tF: taylor(F, xi, 0, 2)
\end{maxima}

{\color{MonVertF}  
\begin{equation*}
  F =
  \begin{maxima}
    tex(tF)
  \end{maxima}
\end{equation*}
}
\subsubsection*{Simplified expression}
\begin{equation}
  \label{eq:SimpF}
  \color{red}
  F = \exp\{-e^{-z}\} \left\{ 1 - e^{-z} \,\frac{z^2}{2} \xi +
    e^{-z} \frac{z^3}{24} \,\left[ 8 - 3 (1 - e^{-z}) z \right] \xi^2 \right\} +
  o(\xi^2)
\end{equation}

\subsection{First-order derivatives: expressions}

\subsubsection*{Scaling}
\begin{equation}
  \color{red}
  \label{eq:ScaleDer1F}
  \begin{aligned}
    \frac{\partial F}{\partial \mu}
    &= \frac{-1}{\sigma} \,\frac{\partial F}{\partial z}\\
    \frac{\partial F}{\partial \sigma}
    &= \frac{-z}{\sigma} \,\frac{\partial F}{\partial z}
  \end{aligned}
\end{equation}

\subsubsection*{Raw expressions}

\begin{maxima}
  Fz: diff(F, z, 1)
\end{maxima}%
\begin{maxima}
  Fxi: diff(F, xi, 1)
\end{maxima}%

{\color{MonVertF}
\begin{align*}
  \frac{\partial F}{\partial z}
  &=
    \begin{maxima}
       tex(Fz)
    \end{maxima} \\
  \frac{\partial F}{\partial \xi}
  &=
    \begin{maxima}
       tex(Fxi)
    \end{maxima} \\
\end{align*}
}
\subsubsection*{Simplified  expressions}
\begin{equation}
  \label{eq:Der1F}
  \color{red}
  \begin{aligned}
    \frac{\partial F}{\partial z}
    &= F W \, \frac{1}{V} \\
    \frac{\partial F}{\partial \xi}
    &= - F W T
  \end{aligned}
\end{equation}

\subsection{First-order derivatives: Taylor expansions
  for $\xi \approx 0$}

\subsubsection*{Raw  expressions}

\begin{maxima}
  tFz: taylor(Fz, xi, 0, 1)
\end{maxima}%
\begin{maxima}
  tFxi: taylor(Fxi, xi, 0, 1)
\end{maxima}%
{\color{MonVertF}
\begin{align*}
  \frac{\partial F}{\partial z}
  &=
    \begin{maxima}
       tex(tFz)
    \end{maxima} \\
  \frac{\partial F}{\partial \xi}
  &=
    \begin{maxima}
       tex(tFxi)
    \end{maxima} \\
\end{align*}}

\subsubsection*{Simplified expressions}
With {\color{red}$F^\star := \exp\{-e^{-z} - z\}$} which is the value
of the standard Gumbel distribution function at $z$,

\begin{equation}
  \color{red}
  \label{eq:SimpDer1F}
  \begin{aligned}
    \frac{\partial F}{\partial z}
    &= F^\star \left\{ 1 + \frac{z}{2} \left[ -2 + (1 -e^{-z}) z \right] \xi \right\}
      + o(\xi) \\
  \frac{\partial F}{\partial \xi}
  &= F^\star
    \left\{-\frac{z^2}{2}
    + \frac{z^3}{12} \left[8 - 3 \, (1 -e^{-z}) z \right] \xi \right\}
    + o(\xi)
  \end{aligned}
\end{equation}

\subsection{Second-order derivatives}

\textbf{CAUTION} The second-order derivatives has not yet been implemented and
the formula have not yet been fully checked.

\subsubsection*{Scaling}

\begin{equation}
  \label{eq:ScalDer2F}
  \color{red}
  \begin{aligned}
    \frac{\partial^2 F}{\partial \mu^2}
    &= \frac{1}{\sigma^2} \,
      \frac{\partial^2 F}{\partial z^2} \\
    \frac{\partial^2 F}{\partial \mu \partial \sigma}
    &= \frac{1}{\sigma^2} \,
      \left\{
      \frac{\partial F}{\partial z} +
      z \frac{\partial^2 F}{\partial z^2}
      \right\} \\
    \frac{\partial^2 F}{\partial \mu \partial \xi}
    &= \frac{-1}{\sigma} \,
      \frac{\partial^2 F}{\partial z \partial \xi}\\
    \frac{\partial^2 F}{\partial \sigma^2}
    &= \frac{z}{\sigma^2} \,
      \left\{2 \, \frac{\partial F}{\partial z} + z
      \frac{\partial^2 F}{\partial z^2} \right\} \\
    \frac{\partial^2 F}{\partial \sigma \partial \xi}
    &= \frac{-z}{\sigma} \,
      \frac{\partial^2 F}{\partial z \partial \xi} 
  \end{aligned}
\end{equation}

\begin{maxima}
 Fzz: diff(F, z, 2)
\end{maxima}%

\begin{maxima}
  Fzxi: diff(F, z, 1, xi, 1)
\end{maxima}%

\begin{maxima}
  Fxixi: diff(F, xi, 2) 
\end{maxima}

\begin{landscape}
\subsubsection*{Raw expressions}
  \color{MonVertF}
\begin{align*}
  \frac{\partial^2 F}{\partial z^2}
  &=
    \begin{maxima}
       tex(Fzz)
     \end{maxima} \\
  \frac{\partial^2 F}{\partial z \partial \xi}
  &=
    \begin{maxima}
      tex(Fzxi)
    \end{maxima} \\
  \frac{\partial^2 F}{\partial \xi^2}
  &=
    \begin{maxima}
       tex(Fxixi)
    \end{maxima} \\
\end{align*}
\end{landscape}


\subsubsection*{Simplified expressions}

Using the auxiliary variables of~(\ref{eq:AuxVar})
\begin{equation}
  \color{red}
  \begin{aligned}
    \frac{\partial^2 F}{\partial z^2}
    &= \frac{F W}{V^2} \,\left\{-[1 + \xi]  + W \right\} \\
    \frac{\partial^2 F}{\partial z \partial \xi}
    &= \frac{F W}{V}  \, \left\{ T \left[1 - W \right]
      - \frac{z}{V} \right\}\\
    \frac{\partial^2 F}{\partial \xi^2}
    &= F W \left\{ - T^2 \left[1 - W \right] + R \right\}
  \end{aligned}
\end{equation}
Note that $W^{1+ 2\xi}= W/V^2$ and the second form is faster to
evaluate.

\subsection{Second-order derivatives: limits for $\xi \to 0$}
%%========================================================

Here are the limits for $\xi \to 0$

\begin{align*}
  \lim_{\xi \to 0} \frac{\partial^2 F}{\partial z^2}
  &=
    \begin{maxima}
      tex(limit (Fzz, xi, 0))
    \end{maxima}
  \\
  \lim_{\xi \to 0} \frac{\partial^2 F}{\partial z \partial \xi}
  &=
    \begin{maxima}
      tex(limit (Fzxi, xi, 0))
    \end{maxima}
  \\
  \lim_{\xi \to 0} \frac{\partial^2 F}{\partial^2 \xi^2}
  &= -{{\left(z^4\,\left(3\,e^{e^ {- z }\,\left(2\,z\,e^{z}+1\right)}-3
 \,e^{e^ {- z }\,\left(z\,e^{z}+1\right)}\right)-8\,z^3\,e^{e^ {- z }
 \,\left(2\,z\,e^{z}+1\right)}\right)\,e^{-e^ {- z }\,\left(2\,z\,e^{
 z}+1\right)-e^ {- z }\,\left(z\,e^{z}+1\right)}}\over{12}}
\end{align*}
It seems that Maxima can not find the limit of
$\partial^2 F/\partial \xi^2$ in a non interactive mode so the value
above was obtained by a paste-and-copy of the result obtained in an
interactive session.

\subsubsection*{Simplified expressions}

With $F^\star := \exp\{-e^{-z} - z\}$ (Gumbel distribution value at $z$), 

\begin{equation}
  \color{red}
  \label{eq:SimpDer2F}
  \begin{aligned}
    \frac{\partial^2 F}{\partial z^2} 
    &= F^\star \left[ e^{-z} - 1  \right] + o(1) \\
    \frac{\partial^2 F}{\partial z \partial \xi}
    &= \frac{1}{2} \, F^\star \left[ z^2 \left(1 - e^{-z} \right) - 2 z  \right] +
      o(1)\\
    \frac{\partial^2 F}{\partial \xi^2}
    &= \frac{1}{12} \, F^\star z^3 \left[ 8 - 3 \left(1 - e^{-z}\right) z \right]+ o(1)
  \end{aligned}
\end{equation}


\section{Quantile or return level}

\subsection{Expression}

%-----------------------------------------------------------------------

With $A:= -\log p$, the quantile $\rho := q_{\text{GEV}}(p)$ is given by
\begin{maxima}
  rho: mu - sigma * (1 -  A^(-xi)) / xi
\end{maxima}


%%
\begin{equation}
  \label{eq:DefRho}
  \color{red}
  \rho= 
  \begin{maxima}
    tex(rho)
  \end{maxima}
\end{equation}

\subsection{Scaling and auxiliary variables}

$$
   \rho = \mu + \sigma \rho^\star
$$

\begin{equation}
  \rho^\star := \frac{ A^{-\xi} - 1}{\xi} 
\end{equation}

\subsection{Taylor expansion for $\xi \approx 0$}

\begin{maxima}
  trho: taylor(rho, xi, 0, 2)
\end{maxima}
{\color{MonVertF}
\begin{equation*}
  \rho = 
  \begin{maxima}
    tex(trho)
  \end{maxima}
\end{equation*}}

\subsection{First-order derivatives: expressions}
%%========================================================
%%
\begin{maxima}
   rhomu: diff(rho, mu, 1)
\end{maxima}
%%
\begin{maxima}
  rhosigma: diff(rho, sigma, 1)
\end{maxima}
%%
\begin{maxima}
  rhoxi: diff(rho, xi, 1)
\end{maxima}

When $\xi \neq 0$ we have
{\color{MonVertF}
\begin{align*}
\frac{\partial \rho}{\partial \mu} &=
\begin{maxima}
 tex(rhomu)
\end{maxima}
\\
\frac{\partial \rho}{\partial \sigma}  &=
\begin{maxima}
 tex(rhosigma)
\end{maxima}
\\
\frac{\partial \rho}{\partial \xi} &=
\begin{maxima}
 tex(rhoxi)
\end{maxima}
\end{align*}
}

{\color{MonVertF}
\begin{align*}
\lim_{\xi \to 0} \frac{\partial \rho}{\partial \mu} &= 
\begin{maxima}
 tex(limit(rhomu, xi, 0))
\end{maxima}
\\
\lim_{\xi \to 0} \frac{\partial \rho}{\partial \sigma} &= 
\begin{maxima}
 tex(limit(rhosigma, xi, 0))
\end{maxima}
\\
\lim_{\xi \to 0} \frac{\partial \rho}{\partial \xi} &= 
\begin{maxima}
 tex(limit(rhoxi, xi, 0))
\end{maxima}
\end{align*}
}

\subsection{First-order derivatives: Taylor expansion for $\xi \approx 0$}
{\color{MonVertF}
\begin{maxima}
  trhomu: taylor(rhomu, xi, 0, 2)
\end{maxima}
\begin{maxima}
  trhosigma: taylor(rhosigma, xi, 0, 1)
\end{maxima}
\begin{maxima}
  trhoxi: taylor(rhoxi, xi, 0, 1)
\end{maxima}
}

{\color{MonVertF}
\begin{align*}
  \frac{\partial \rho}{\partial \mu}
  &=
  \begin{maxima}
    tex(trhomu)
  \end{maxima} \\
  \frac{\partial \rho}{\partial \sigma}
  &=
  \begin{maxima}
    tex(trhosigma)
  \end{maxima} \\
  \frac{\partial \rho}{\partial \xi}
  &=
  \begin{maxima}
    tex(trhoxi)
  \end{maxima}
\end{align*}
}

\subsection{Second-order derivatives: expressions}
%% ========================================================

\subsubsection*{Scaling}
%%----------------------
Since $\rho$ is a linear function of $\mu$ and $\sigma$, we
consider only the standardized return level
\begin{equation*}%
  \begin{maxima}%
    rhoStar: - (1 - A^(-xi)) / xi
  \end{maxima}%
\end{equation*}%
%% 
\begin{equation}
  \label{eq:ScaleRho}
  \color{red}
  \rho^\star=
  \begin{maxima}
    tex(rhoStar)
  \end{maxima}
\end{equation}
so that $\rho = \mu + \sigma \rho^\star$.

\subsubsection*{Raw expressions}
%%
\begin{maxima}
  rhoStarxi:  diff(rhoStar, xi, 1)
\end{maxima}

{\color{MonVertF}
\[
\frac{\partial \rho^\star}{\partial \xi} = 
\begin{maxima}
  tex(rhoStarxi)
\end{maxima}
\]}

%% ========================================================

\begin{maxima}
  rhoStarxixi:  diff(rhoStar, xi, 2)
\end{maxima}

{\color{MonVertF}
\[
\frac{\partial^2 \rho^\star}{\partial \xi^2} = 
\begin{maxima}
  tex(rhoStarxixi)
\end{maxima}
\]}

\subsubsection*{Simplified expression}
\begin{equation}
  \label{eq:SimpDer1RhoStar}
  \color{red}
  \frac{\partial \rho^\star}{\partial \xi} =
  - \frac{1}{\xi}\, \left[ \rho^\star + \log A \right] - \rho^\star \, \log A
\end{equation}


\begin{equation}
  \label{eq:SimpDer2RhoStar}
  \color{red}
  \frac{\partial^2 \rho^\star}{\partial \xi^2} = \frac{1}{\xi^2} \,
  \left[ \rho^\star + \log A \right]
  - \frac{\partial \rho^\star}{\partial \xi} \left[ \log A +
    \frac{1}{\xi} \right]
\end{equation}
 
$$
\frac{\partial \rho}{\partial \boldsymbol{\theta}\partial \boldsymbol{\theta}^\top} =
\begin{bmatrix}
  0 & 0 & 0 \\
  0 & 0 & \frac{\partial \rho^\star}{\partial \xi}\\
  0 & \frac{\partial \rho^\star}{\partial \xi} &
  \sigma \frac{\partial^2 \rho^\star}{\partial \xi^2}
\end{bmatrix}
$$

\subsection{Second-order derivatives: limit for $\xi \to 0$}
%%========================================================
Here are the limits for $\xi \to 0$

\begin{equation}
  \color{MonVertF}
  \label{eq:LimD12RhoStar}
  \begin{aligned}
    \lim_{\xi \to 0} \frac{\partial \rho^\star}{\partial \xi}
    &=
      \begin{maxima}
        tex(limit(rhoStarxi, xi, 0))
      \end{maxima}
    \\
    \lim_{\xi \to 0} \frac{\partial^2 \rho^\star}{\partial \xi^2}
    &=
      \begin{maxima}
        tex(limit(rhoStarxixi, xi, 0))
      \end{maxima}
  \end{aligned}
\end{equation}
