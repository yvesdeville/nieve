
%% \chapter{GEV}

\section{Scope}

We consider a ``quantity'' related to the GEV distribution: the
log-density $\log f(x;\,\boldsymbol{\theta})$, the distribution
$F(q;\,\boldsymbol{\theta})$ and or quantile
$q(p;\,\boldsymbol{\theta})$. The vector $\boldsymbol{\theta}$ is
formed by the trhee elements: location~$\mu$, scale~$\sigma$ and
shape~$\xi$. We want to have a closed-form expression for each
quantity and its first-order derivative w.r.t. $\boldsymbol{\theta}$
and possibly the second-order derivative.

As is weel-known, the expressions defining the quantities take
depending on $\xi = 0$ or $\xi \neq 0$. The expressions for
$\xi \neq 0$ are usable for very small $\xi$, say $\xi \approx 1e-7$,
but will show some inacuracy when $\xi$ is very small. We will use
specific expression for $|\xi| < \epsilon$ where $\epsilon$ is a very small
positive real. These expressions are derived from Taylor expansions
at $\xi = 0$. More precisely we use

\begin{itemize}
\item Second order Taylor approximation for the quantity,
\item First  order Taylor approximation for the derivative of the quantity,
\item Zero order Taylor approximation for the 2-nd order
  derivative of the quantity.
\end{itemize}
So the derivatives up to order two will be continuous and moreover
they will be consistent. For instance the approximation of the 1-st
order derivative is the derivative of the approximation at $\xi = 0$.

As a rule, we implement the formulas given {\color{red}in red} but we
give the formulas given by \textbf{maxima} {\color{MonVertF} in
  green}. 

This document was generated by using the
\href{https://maxima.sourceforge.io/Maxima}{Maxima Computer Algebra System}
and the \textbf{maxiplot} package for \LaTeX{}.


\section{Log-density $\log f$}

\subsection{Expression}

\begin{maxima}
  V: 1 + xi * z
\end{maxima}
\begin{maxima}
  U: (1 + \xi - V^(-1/xi)) / sigma / V
\end{maxima}

Define $z$ by
$$
z := \frac{y - \mu}{\sigma}
$$
%-----------------------------------------------------------------------
and then $V$ and $U$ by 
\[
V := 
\begin{maxima}
tex(V)
\end{maxima}, \qquad U:= \frac{1}{\sigma V}\,\left[ 1 + \xi - V^{-1/\xi}\right]. 
\]
%%-----------------------------------------------------------------------
\begin{maxima}
  g: -log(sigma) - (1 / xi + 1) * log(V) - V^(-1 / xi)
\end{maxima}
%
\begin{maxima}
  gz: diff(g, z, 1)
\end{maxima}%
%% 
\begin{maxima}
  gxi: diff(g, xi, 1)
\end{maxima}%
%%
% -----------------------------------------------------------------------

The log-likelihood is defined by
\[
\log f = 
\begin{maxima}
  tex(g)
\end{maxima}
\]

\begin{maxima}
  g: -log(sigma) - (1 / xi + 1) * log(V) - V^(-1 / xi)
\end{maxima}

% \subsection{Derivatives}
% %%=======================================================

% \begin{align*}
%   \frac{\partial \log f}{\partial z}
%   &=
%     \begin{maxima}
%       tex(gz)
%     \end{maxima}
%   \\
%   \frac{\partial \log f}{\partial \xi}
%   &=
%     \begin{maxima}
%       tex(gxi)
%     \end{maxima}
% \end{align*}


\subsection{Taylor expansion at $\xi = 0$}
%%=======================================================

\begin{maxima}
  txi: taylor(g, xi, 0, 2)
\end{maxima}
{\color{MonVertF}
\[
  \log f = 
  \begin{maxima}
    tex(txi)
  \end{maxima}
\]
}

which can be simplified as
{\color{red}
$$
\log f = -\left[ \log \sigma + z + e^{-z} \right] + \frac{1}{2}\,
\left[ (1 -  e^{-z}) z -2  \right]\,z \xi - \frac{1}{24} \,
\left[ 3 z^2 e^{-z} + 8 z (1 - e^{-z}) - 12 \right] z^2 \xi^2
$$}

\subsection{First-order derivatives: expression}
%%=======================================================

\subsubsection*{Remark: scaling}
%%-----------------------------------
In order to get quite simple expressions, we remark that $\log f$
depends on $\mu$ only through $z$, so
$$
\frac{\partial \log f}{\partial \mu} = \frac{\partial \log f}{\partial z} \,
   \dfrac{\partial z}{\partial \mu} = -\frac{1}{\sigma} \, \frac{\partial \log f}{\partial z}
$$ 
and up to the term $-\log \sigma$ in $\log f$, the same is true for
$\sigma$ so
$$
\frac{\partial \log f}{\partial \sigma} =
-\frac{1}{\sigma} + \frac{\partial \log f}{\partial z} \, \dfrac{\partial z}{\partial \sigma} =
-\frac{1}{\sigma} -\frac{z}{\sigma} \, \frac{\partial \log f}{\partial z}.
$$ 
So it is simpler to compute the derivative w.r.t. $z$ and then find the
derivatives w.r.t. $\mu$ and $\sigma$.

\subsubsection*{Raw expressions}
%%-----------------------------------
Here are the raw expressions found by Maxima
{\color{MonVertF}
  \begin{align*}
  \frac{\partial \log f}{\partial z}
  &=
    \begin{maxima}
      tex(gz)
    \end{maxima}
  \\
  \frac{\partial \log f}{\partial \xi}
  &=
    \begin{maxima}
      tex(gxi)
    \end{maxima}
  \end{align*}
}

\subsubsection*{Simplified expressions}

We use the following simplifications
\begin{align*} 
  \frac{\partial \log f}{\partial z}
  &= - \sigma U, %% V^{-(\xi + 1)/\xi} - (\xi + 1) V^{-1},
  \\
  \frac{\partial \log f}{\partial \xi}
  &= \frac{1}{\xi^2} \, [1 - V^{-1/\xi}]  \, \log V - z \, U \frac{\sigma}{\xi}.
\end{align*}
so eventually

{\color{red}
\begin{align*} 
  \frac{\partial \log f}{\partial \mu}
  &= U,
  \\
  \frac{\partial \log f}{\partial \sigma}
  &= \frac{-1}{\sigma} + z U,
  \\
  \frac{\partial \log f}{\partial \xi}
  &= \frac{1}{\xi^2} \, [1 - V^{-1/\xi}]  \, \log V - z \, U \frac{\sigma}{\xi}.
\end{align*}
}

\subsection{First-order derivatives: Taylor expansions for $\xi \approx 0$}
\begin{maxima}
  tU: taylor(U, xi, 0, 1)
\end{maxima}
\begin{maxima}
  tgz: taylor(gz, xi, 0, 1)
\end{maxima}
\begin{maxima}
  tgxi: taylor(gxi, xi, 0, 1)
\end{maxima}
{\color{MonVertF}
$$
U =
\begin{maxima}
  tex(tU)
\end{maxima}
$$
}

{\color{MonVertF}
\begin{align*}
  \frac{\partial \log f}{\partial z} &= 
  \begin{maxima}
    tex(tgz)
  \end{maxima} \\
  \frac{\partial \log f}{\partial \xi} &= 
  \begin{maxima}
    tex(tgxi)
  \end{maxima}
\end{align*}
}

{\color{red}
\begin{align*}
\frac{\partial \log f}{\partial z} &= -\left[ 1 - e^{-z} \right]
- \frac{1}{2}\, \left[2 -2 (1 - e^{-z}) z -  e^{-z} z^2 \right] \, \xi + o(\xi)  \\
\frac{\partial \log f}{\partial \xi} &= -\frac{1}{2}\,\left[ 2 z - (1 - e^{-z}) z^2 \right]
+  \frac{1}{12}\, \left[12 - 8 (1 - e^{-z}) z - 3z^2\right] \, z^2 \xi + o(\xi) 
\end{align*}
}

\subsubsection*{Limits for $\xi \to 0$}
%%========================================================
Here are the limits for $\xi \to 0$
{\color{MonVertF}
  \begin{align*}
  \lim_{\xi \to 0} \frac{\partial \log f}{\partial z}
  &=
    \begin{maxima}
      tex(limit (gz, xi, 0))
    \end{maxima}
  \\
  \lim_{\xi \to 0} \frac{\partial \log f}{\partial \xi}
  &=
    \begin{maxima}
      tex(limit (gxi, xi, 0))
    \end{maxima}
\end{align*}}
that is
{\color{red}
  \begin{align*}
  \lim_{\xi \to 0} \frac{\partial \log f}{\partial z}
  &= 1 - e^{-z}
  \\
  \lim_{\xi \to 0} \frac{\partial \log f}{\partial \xi}
  &= -\frac{z}{2} \,\left[2 - (1 - e^{-z}) z \right] 
\end{align*}}



\subsection{Second-order derivatives: expressions}
%% ============================================
\subsubsection*{Scaling}
%% -----------------------------------
{\color{red}
\begin{align*}
  \frac{\partial^2 \log f}{\partial \mu^2}
  &= \frac{1}{\sigma^2} \,
    \frac{\partial^2 \log f}{\partial z^2} \\
  \frac{\partial^2 \log f}{\partial \mu \partial \sigma}
  &= \frac{z}{\sigma^2} \,
    \frac{\partial^2 \log f}{\partial z^2} \\
  \frac{\partial^2 \log f}{\partial \mu \partial \xi}
  &= \frac{-1}{\sigma} \,
    \frac{\partial^2 \log f}{\partial z \partial \xi}\\
  \frac{\partial^2 \log f}{\partial \sigma^2}
  &= \frac{1}{\sigma^2} +
    \frac{z^2}{\sigma^2} \,\frac{\partial \log f}{\partial z}  -
    \frac{z}{\sigma} \,
    \frac{\partial^2 \log f}{\partial z^2}  \\
  \frac{\partial^2 \log f}{\partial \sigma \partial \xi}
  &= \frac{-z}{\sigma} \,
    \frac{\partial^2 \log f}{\partial z \partial \xi} 
\end{align*}
}

\subsubsection*{Raw expressions}
%%-----------------------------------
\begin{maxima}
  gzz: diff(g, z, 2)
\end{maxima}%

\begin{maxima}
  gzzb: letsimp(gzz)
\end{maxima}%

\begin{maxima}
  gzxi: diff(g, z, 1, xi, 1)
\end{maxima}%

\begin{maxima}
  gzxib: letsimp(gzxi)
\end{maxima}%

\begin{maxima}
  gxixi: diff(g, xi, 2)
\end{maxima}%

\begin{maxima}
  gxixib: letsimp(gxixi)
\end{maxima}%

% %% \begin{landscape}
% % %   -------------------------------------
{\color{MonVertF}
\begin{align*}
  \frac{\partial^2 \log f}{\partial z^2}
  &=
  \begin{maxima}
    tex(gzzb)
  \end{maxima}\\
  \frac{\partial^2 \log f}{\partial z \partial \xi}
  &=
  \begin{maxima}
    tex(gzxib)
  \end{maxima}\\
  \frac{\partial^2 \log f}{\partial \xi^2}
  &=
  \begin{maxima}
    tex(gxixib)
  \end{maxima}
\end{align*}
}

\subsubsection*{Simplified expressions}
%%-----------------------------------
{\color{red}
\begin{align*}
  \frac{\partial^2 \log f}{\partial z^2}
  &= - (\xi + 1) \left[V^{-1/\xi} - \xi\right] \times \frac{1}{ V^2}\\
  \frac{\partial^2 \log f}{\partial z \partial \xi}
  &=  V^{-1/\xi -1}
    \left[\frac{1}{\xi^2} \,\log V - \frac{(\xi + 1)}{\xi}\,\frac{z}{V} \right]
    - \frac{1}{V} + (\xi +1) \frac{z}{V^2}   \\
  \frac{\partial^2 \log f}{\partial \xi^2} 
  &= - \left\{ \left[\frac{1}{\xi^2} \, \log V - \frac{1}{\xi}\,\frac{z}{V} \right]^2 +
    \left[ - \frac{2}{\xi^2}\,\log V + \frac{2}{\xi}\,\frac{z}{V}
    + \frac{z^2}{V^2}\right] \times
    \frac{1}{\xi} \right\} \times V^{-1/\xi} - \frac{2}{\xi^3}\,\log V +
    \frac{2}{\xi^2}\,\frac{z}{V} + \frac{\xi +1}{\xi}\, \frac{z^2}{V^2}
\end{align*}
}




% \section{Distribution function $F$}
\subsection{Second-order derivatives: limits for $\xi \to 0$}
%%========================================================
Here are the limits for $\xi \to 0$
{\color{MonVertF}
  \begin{align*}
  \lim_{\xi \to 0} \frac{\partial^2 \log f}{\partial^2 z^2}
  &=
    \begin{maxima}
      tex(limit (gzz, xi, 0))
    \end{maxima}
  \\
  \lim_{\xi \to 0} \frac{\partial^2 \log f}{\partial z \partial \xi}
  &=
    \begin{maxima}
      tex(limit (gzxi, xi, 0))
    \end{maxima}
  \\
  \lim_{\xi \to 0} \frac{\partial^2 \log f}{\partial^2 \xi^2}
   &=
    \begin{maxima}
      tex(limit (gxixi, xi, 0))
    \end{maxima}
\end{align*}
}

{\color{red}
\begin{align*}
  \lim_{\xi \to 0} \frac{\partial^2 \log f}{\partial^2 z^2}
  &= -e^{-z}
  \\
  \lim_{\xi \to 0} \frac{\partial^2 \log f}{\partial z \partial \xi}
  &= \frac{1}{2} \, \left[ -2  - 2 (1 - e^{-z}) z  + e^{-z} z^2\right]  
  \\
  \lim_{\xi \to 0} \frac{\partial^2 \log f}{\partial^2 \xi^2}
   &=
    \frac{z^2}{12} \, \left[12 + 8 (1 - e^{-z}) z -3 e^{-z} z^2 \right]
\end{align*}}

\section{Distribution function $F$}

\subsection{Expression}

\begin{maxima}
  W: V^(-1/xi)
\end{maxima}

Define
$$
W := V^{-1/\xi} = [1 + \xi z]^{-1/\xi}.
$$
Then
$$
 F := \exp\{ - W \}.
$$
For the distribution function we will only consider the first-order derivatives
since the second-order derivatives are not needed in most applications.

\begin{maxima}
  F: exp(-W)
\end{maxima}

\subsection{Taylor expansions for $\xi \approx 0$}
\subsubsection*{Raw expression}

\begin{maxima}
  tF: taylor(F, xi, 0, 2)
\end{maxima}

{\color{MonVertF}  
\begin{equation*}
  F =
  \begin{maxima}
    tex(tF)
  \end{maxima}
\end{equation*}
}
\subsubsection*{Simplified expression}
{\color{red}
$$
F = \exp\{-e^{-z}\} \left\{ 1 - e^{-z} \,\frac{z^2}{2} \xi +
  e^{-z} \frac{z^3}{24} \,\left[ 8 - 3 (1 - e^{-z}) z \right] \xi^2 \right\} +
  o(\xi^2)
$$}

\subsection{First-order derivatives: expressions}

\subsubsection*{Scaling}

\begin{align*}
\frac{\partial F}{\partial \mu} &= \frac{-1}{\sigma} \,\frac{\partial F}{\partial z}\\
\frac{\partial F}{\partial \sigma} &= \frac{z}{\sigma} \,\frac{\partial F}{\partial z}\\
\end{align*}

\subsubsection*{Raw expressions}

\begin{maxima}
  Fz: diff(F, z, 1)
\end{maxima}%
\begin{maxima}
  Fxi: diff(F, xi, 1)
\end{maxima}%

{\color{MonVertF}
\begin{align*}
  \frac{\partial F}{\partial z}
  &=
    \begin{maxima}
       tex(Fz)
    \end{maxima} \\
  \frac{\partial F}{\partial \xi}
  &=
    \begin{maxima}
       tex(Fxi)
    \end{maxima} \\
\end{align*}
}
\subsubsection*{Simplified  expressions}
{\color{red}
\begin{align*}
  \frac{\partial F}{\partial z}
  &= F W \, \frac{1}{V} \\
  \frac{\partial F}{\partial \xi}
  &= - F W  \left\{ \frac{\log V}{\xi^2} - \frac{z}{\xi V} \right\}\\
\end{align*}
}



\subsection{First-order derivatives: Taylor expansions for $\xi \approx 0$}

\subsubsection*{Raw  expressions}

\begin{maxima}
  tFz: taylor(Fz, xi, 0, 1)
\end{maxima}%
\begin{maxima}
  tFxi: taylor(Fxi, xi, 0, 1)
\end{maxima}%
{\color{MonVertF}
\begin{align*}
  \frac{\partial F}{\partial z}
  &=
    \begin{maxima}
       tex(tFz)
    \end{maxima} \\
  \frac{\partial F}{\partial \xi}
  &=
    \begin{maxima}
       tex(tFxi)
    \end{maxima} \\
\end{align*}}

\subsubsection*{Simplified expressions}

{\color{red}
\begin{align*}
  \frac{\partial F}{\partial z}
  &= \exp\{ -z - e^{-z} \} \left\{ 1 + \frac{z}{2} \left[ -2 + (1 -e^{-z}) z \right] \xi \right\} + o(\xi^2) \\
  \frac{\partial F}{\partial \xi}
  &= \exp\{ -z - e^{-z} \}
    \left\{-\frac{z^2}{2}
    + \frac{z^3}{12} \left[8 - 3 (1 -e^{-z}) z \right] \xi \right\}
    + o(\xi^2)
\end{align*}
}

% \subsection{Second-order derivatives}
% \begin{maxima}
%   Fzz: diff(F, z, 2)
% \end{maxima}%
% \begin{maxima}
%   Fzxi: diff(F, z, 1, xi, 1)
% \end{maxima}%

% \begin{maxima}
%   Fxixi: diff(F, xi, 2) 
% \end{maxima}%

% %% \begin{landscape}
% \begin{align*}
%   \frac{\partial^2 F}{\partial z^2}
%   &=
%     \begin{maxima}
%        tex(Fzz)
%      \end{maxima} \\
%   \frac{\partial^2 F}{\partial z \partial \xi}
%   &=
%     \begin{maxima}
%       tex(Fzxi)
%     \end{maxima} \\
%   \frac{\partial^2 F}{\partial \xi^2}
%   &=
%     \begin{maxima}
%        tex(Fxixi)
%     \end{maxima} \\
% \end{align*}
% %% \end{landscape}

% \subsubsection{Limits for $\xi \to 0$}
% %%========================================================
% Here are the limits for $\xi \to 0$
% \begin{align*}
%   \lim_{\xi \to 0} \frac{\partial^2 F}{\partial z^2}
%   &=
%     \begin{maxima}
%       tex(limit (Fzz, xi, 0))
%     \end{maxima}
%   \\
%   \lim_{\xi \to 0} \frac{\partial^2 F}{\partial z \partial \xi}
%   &=
%     \begin{maxima}
%       tex(limit (Fzxi, xi, 0))
%     \end{maxima}
%   \\
%   % \lim_{\xi \to 0} \frac{\partial^2 F}{\partial^2 \xi^2}
%   %  &=
%   %   \begin{maxima}
%   %     tex(limit (Fxixi, xi, 0))
%   %   \end{maxima}
% \end{align*}

\section{Quantile or return level}

\subsection{Expression}

%-----------------------------------------------------------------------

With $A:= -\log p$, the quantile $\rho := q_{\text{GEV}}(p)$ is given by
\begin{maxima}
  rho: mu - sigma * (1 -  A^(-xi)) / xi
\end{maxima}


%%
\[
\rho=
\begin{maxima}
 tex(rho)
\end{maxima}
\]

\subsection{Taylor expansion for $\xi \approx 0$}

\begin{maxima}
  trho: taylor(rho, xi, 0, 2)
\end{maxima}
{\color{MonVertF}
\begin{equation*}
  \rho = 
  \begin{maxima}
    tex(trho)
  \end{maxima}
\end{equation*}}

\subsection{First-order derivatives: expressions}
%%========================================================
%%
\begin{maxima}
   rhomu: diff(rho, mu, 1)
\end{maxima}
%%
\begin{maxima}
  rhosigma: diff(rho, sigma, 1)
\end{maxima}
%%
\begin{maxima}
  rhoxi: diff(rho, xi, 1)
\end{maxima}

When $\xi \neq 0$ we have
{\color{MonVertF}
\begin{align*}
\frac{\partial \rho}{\partial \mu} &=
\begin{maxima}
 tex(rhomu)
\end{maxima}
\\
\frac{\partial \rho}{\partial \sigma}  &=
\begin{maxima}
 tex(rhosigma)
\end{maxima}
\\
\frac{\partial \rho}{\partial \xi} &=
\begin{maxima}
 tex(rhoxi)
\end{maxima}
\end{align*}
}

{\color{MonVertF}
\begin{align*}
\lim_{\xi \to 0} \frac{\partial \rho}{\partial \mu} &= 
\begin{maxima}
 tex(limit(rhomu, xi, 0))
\end{maxima}
\\
\lim_{\xi \to 0} \frac{\partial \rho}{\partial \sigma} &= 
\begin{maxima}
 tex(limit(rhosigma, xi, 0))
\end{maxima}
\\
\lim_{\xi \to 0} \frac{\partial \rho}{\partial \xi} &= 
\begin{maxima}
 tex(limit(rhoxi, xi, 0))
\end{maxima}
\end{align*}
}

\subsection{First-order derivatives: Taylor expansion for $\xi \approx 0$}
{\color{MonVertF}
\begin{maxima}
  trhomu: taylor(rhomu, xi, 0, 2)
\end{maxima}
\begin{maxima}
  trhosigma: taylor(rhosigma, xi, 0, 1)
\end{maxima}
\begin{maxima}
  trhoxi: taylor(rhoxi, xi, 0, 1)
\end{maxima}
}

{\color{MonVertF}
\begin{align*}
  \frac{\partial \rho}{\partial \mu}
  &=
  \begin{maxima}
    tex(trhomu)
  \end{maxima} \\
  \frac{\partial \rho}{\partial \sigma}
  &=
  \begin{maxima}
    tex(trhosigma)
  \end{maxima} \\
  \frac{\partial \rho}{\partial \xi}
  &=
  \begin{maxima}
    tex(trhoxi)
  \end{maxima}
\end{align*}
}

\subsection{Second-order derivatives: expressions}
%% ========================================================

\subsubsection*{Scaling}
%%----------------------
Since $\rho$ is a linear function of $\mu$ and $\sigma$, we
consider only the standardised return level
\begin{maxima}
  rhoStar: - (1 - A^(-xi)) / xi
\end{maxima}
%% 
\[
\rho^\star=
\begin{maxima}
  tex(rhoStar)
\end{maxima}
\]
so that $\rho = \mu + \sigma \rho^\star$.

\subsubsection*{Raw expressions}
%%
\begin{maxima}
  rhoStarxi:  diff(rhoStar, xi, 1)
\end{maxima}

{\color{MonVertF}
\[
\frac{\partial \rho^\star}{\partial \xi} = 
\begin{maxima}
  tex(rhoStarxi)
\end{maxima}
\]}

%% ========================================================

\begin{maxima}
  rhoStarxixi:  diff(rhoStar, xi, 2)
\end{maxima}

{\color{MonVertF}
\[
\frac{\partial^2 \rho^\star}{\partial \xi^2} = 
\begin{maxima}
  tex(rhoStarxixi)
\end{maxima}
\]}

\subsubsection*{Simplified expression}
{\color{red}
$$
\frac{\partial \rho^\star}{\partial \xi} =
- \frac{1}{\xi}\, \left[ \rho^\star + \log A \right] - \rho^\star \, \log A
$$
}

{\color{red}
$$
\frac{\partial^2 \rho^\star}{\partial \xi^2} = \frac{1}{\xi^2} \,
\left[ \rho^\star + \log A \right]
  - \frac{\partial \rho^\star}{\partial \xi} \left[ \log A + \frac{1}{\xi} \right]
$$
}

$$
\frac{\partial \rho}{\partial \boldsymbol{\theta}\partial \boldsymbol{\theta}^\top} =
\begin{bmatrix}
  0 & 0 & 0 \\
  0 & 0 & \frac{\partial \rho^\star}{\partial \xi}\\
  0 & \frac{\partial \rho^\star}{\partial \xi} &
  \sigma \frac{\partial^2 \rho^\star}{\partial \xi^2}
\end{bmatrix}
$$

\subsection{Second-order derivatives: limit for $\xi \to 0$}
%%========================================================
Here are the limits for $\xi \to 0$
{\color{MonVertF}
\begin{align*}
  \lim_{\xi \to 0} \frac{\partial \rho^\star}{\partial \xi}
  &=
    \begin{maxima}
      tex(limit(rhoStarxi, xi, 0))
    \end{maxima}
  \\
   \lim_{\xi \to 0} \frac{\partial^2 \rho^\star}{\partial \xi^2}
  &=
    \begin{maxima}
      tex(limit(rhoStarxixi, xi, 0))
    \end{maxima}
\end{align*}}
