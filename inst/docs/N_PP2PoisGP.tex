%% \chapter{Poisson-GP to NHPP transformations}

\section{The PP to Poisson-GP transformation}

\subsection{Expression}

\begin{maxima}
  z_0: (u - mu_0) / sigma_0
\end{maxima}

\begin{maxima}
  C: 1 + xi_0 * z_0
\end{maxima}
With $w$ being the reference block duration, 
$$
   z_0 := \frac{u - \mu_0}{\sigma_0}, \qquad C:= 1 + \xi_0 z_0
$$

{\color{MonVertF}
\begin{maxima}
  lambda: C^(-1 /xi_0) / w
\end{maxima}
\begin{maxima}
  sigma: C * sigma_0
\end{maxima}
\begin{maxima}
  xi: xi_0
\end{maxima}
}
The Poisson-GP parameter vector
$\boldsymbol{\theta} := [\lambda, \, \sigma, \, \xi]$ relates to the
vector $\boldsymbol{\theta}_0 := [\mu_0, \, \sigma_0, \, \xi_0]$ of PP
parameters according to {\color{MonVertF}
\begin{align*}
  \lambda
  &=
    \begin{maxima}
      tex(lambda)
    \end{maxima}\\
  \sigma
  & =
    \begin{maxima}
      tex(sigma)
    \end{maxima}\\   
  \xi &= \xi_0.
\end{align*}}%
In simpler form
{\color{red}
\begin{align*}
  \lambda
  &= C^{-1/\xi_0} w^{-1}\\
  \sigma
  & = C \sigma_0\\   
  \xi &= \xi_0
\end{align*}}%
Remind that these expressions can only be used when the threshold $u$
is in the support of the GEV distribution with parameters
$\boldsymbol{\theta}_0 $.

{\footnotesize
\begin{maxima}
  tlambda: taylor(lambda, xi_0, 0, 2)
\end{maxima}
\begin{maxima}
  tsigma: taylor(sigma, xi_0, 0, 2)
\end{maxima}
\begin{maxima}
  txi: taylor(xi, xi_0, 0, 2)
\end{maxima}
}

\subsection{Taylor approximation for $\xi \approx 0$}

\subsubsection*{Raw expressions}

%% \begin{landscape}
{\footnotesize \color{MonVertF}
\begin{align*}
\lambda &= 
\begin{maxima}
  tex(tlambda)
\end{maxima} \\
\sigma &= 
\begin{maxima}
  tex(tsigma)
\end{maxima}\\
  \xi &= 
\begin{maxima}
  tex(txi)
\end{maxima}
\end{align*}
}
%% \end{landscape}

\subsubsection*{Simplified expressions}

{\color{red}
\begin{align*}
  \lambda &= e^{-z_0} w^{-1} \,  \left\{ 1 + z_0^2 \, \xi -
            \frac{z_0^3}{24}\, \left[ 8 - 3 z_0 \right] \xi_0^2 \right\}
            + o(\xi_0^2)\\
  \sigma &= \sigma_0 \,  \left[ 1 + z_0 \, \xi  \right] + o(\xi_0^2)\\
  \xi &= \xi_0
\end{align*}
}



\section{First-order derivatives for $\lambda$: expressions}


\subsection*{Raw expressions}

\begin{maxima}
  dlambdadmu_0: diff(lambda, mu_0, 1)
\end{maxima}
\begin{maxima}
  dlambdadsigma_0: diff(lambda, sigma_0, 1)
\end{maxima}
\begin{maxima}
  dlambdadxi_0: diff(lambda, xi_0, 1)
\end{maxima}

{\color{MonVertF}
\begin{align*}
  \frac{\partial{\lambda}}{\partial \mu_0}
  &=
\begin{maxima}
  tex(dlambdadmu_0)
\end{maxima}\\
 \frac{\partial{\lambda}}{\partial \sigma_0}
  &=
\begin{maxima}
  tex(dlambdadxi_0)
\end{maxima}\\
\frac{\partial{\lambda}}{\partial \xi_0}
  &=
\begin{maxima}
  tex(dlambdadxi_0)
\end{maxima}
\end{align*}
}

\subsection*{Simplified expressions}

{\color{red}
\begin{align*}
  \frac{\partial{\lambda}}{\partial \mu_0}
  &= \frac{\lambda}{\sigma^\star C} \\
 \frac{\partial{\lambda}}{\partial \sigma_0}
  &=
 \frac{z \lambda}{\sigma^\star C}\\
\frac{\partial{\lambda}}{\partial \xi_0}
  &=  \frac{\lambda}{\xi_0^2} \, \left[ \log C - \xi_0 \frac{z_0}{C} \right]\\
\end{align*}
}

\section{First-order derivatives of $\lambda$: Taylor approximation}


\subsection*{Raw expressions}

\begin{maxima}
  tdlambdadmu_0: taylor(dlambdadmu_0, xi, 0, 1)
\end{maxima}
\begin{maxima}
  tdlambdadsigma_0: taylor(dlambdadsigma_0, xi, 0, 1)
\end{maxima}
\begin{maxima}
  tdlambdadxi_0: taylor(dlambdadxi_0, xi, 0, 1)
\end{maxima}

{\color{MonVertF} \footnotesize
\begin{align*}
  \frac{\partial{\lambda}}{\partial \mu_0}
  &=
\begin{maxima}
  tex(tdlambdadmu_0)
\end{maxima}\\
 \frac{\partial{\lambda}}{\partial \sigma_0}
  &=
\begin{maxima}
  tex(tdlambdadsigma_0)
\end{maxima}\\
 \frac{\partial{\lambda}}{\partial \xi_0}
  &=
\begin{maxima}
  tex(tdlambdadxi_0)
\end{maxima}
\end{align*}
}

\subsection*{Simplified expressions}

{\color{red} 
\begin{align*}
  \frac{\partial{\lambda}}{\partial \mu_0}
  &= \frac{1}{\sigma_0} \, e^{-z_0} w^{-1} \,
    \left\{ 1 + \frac{z_0}{2} \,
    \left[z_0 - 2\right] \, \xi_0 \right\} + o(\xi_0)
  \\
 \frac{\partial{\lambda}}{\partial \sigma_0}
  &= \frac{z_0}{\sigma_0} \, e^{-z_0} w^{-1} \,
    \left\{ 1 + \frac{z_0}{2} \,
    \left[z_0 - 2\right] \, \xi_0 \right\} + o(\xi_0)
  \\
 \frac{\partial{\lambda}}{\partial \xi_0}
  &= \frac{z_0^2}{2} \, e^{-z_0} w^{-1} \,
     \left\{ 1 + \frac{z_0}{12} \,
    \left[3 z_0 - 8\right] \, \xi_0 \right\} + o(\xi_0)
  \\
\end{align*}
}

\section{First-order derivatives for $\sigma$: expressions}

\subsection*{Raw expressions}

\begin{maxima}
  dsigmadmu_0: diff(sigma, mu_0, 1)
\end{maxima}
\begin{maxima}
  dsigmadsigma_0: diff(sigma, sigma_0, 1)
\end{maxima}
\begin{maxima}
  dsigmadxi_0: diff(sigma, xi_0, 1)
\end{maxima}

\begin{maxima}
  tdsigmadmu_0: taylor(dsigmadmu_0, xi, 0, 1)
\end{maxima}
\begin{maxima}
  tdsigmadsigma_0: taylor(dsigmadsigma_0, xi, 0, 1)
\end{maxima}
\begin{maxima}
  tdsigmadxi_0: taylor(dsigmadxi_0, xi, 0, 1)
\end{maxima}

{\color{MonVertF} 
\begin{align*}
  \frac{\partial{\sigma}}{\partial \mu_0}
  &=
\begin{maxima}
  tex(dsigmadmu_0)
\end{maxima}\\
 \frac{\partial{\sigma}}{\partial \sigma_0}
  &=
\begin{maxima}
  tex(dsigmadsigma_0)
\end{maxima}\\
 \frac{\partial{\sigma}}{\partial \xi_0}
  &=
\begin{maxima}
  tex(dsigmadxi_0)
\end{maxima}
\end{align*}
}

\section{First-order derivatives for $\sigma$:
  Taylor approximation}

Unneeded (see previous section).

% {\color{MonVertF} 
% \begin{align*}
%   \frac{\partial{\sigma}}{\partial \mu_0}
%   &=
% \begin{maxima}
%   tex(tdsigmadmu_0)
% \end{maxima}\\
%  \frac{\partial{\sigma}}{\partial \sigma_0}
%   &=
% \begin{maxima}
%   tex(tdsigmadsigma_0)
% \end{maxima}\\
%  \frac{\partial{\sigma}}{\partial \xi_0}
%   &=
% \begin{maxima}
%   tex(tdsigmadxi_0)
% \end{maxima}
% \end{align*}
% }

% \subsubsection*{Simplified  expressions}

% {\color{red} 
% \begin{align*}
%   \frac{\partial{\sigma}}{\partial \mu_0}
%   &= -\xi_0 + o(\xi_0)\\
%  \frac{\partial{\sigma}}{\partial \sigma_0}
%   &= 1 + o(\xi_0) \\
%  \frac{\partial{\sigma}}{\partial \xi_0}
%   &= \sigma_0 z_0 +o(\xi_0)
% \end{align*}
% }

\section{First-order derivatives for $\xi$: expressions}

\subsection*{Simplified expressions}
{\color{red} 
\begin{align*}
  \frac{\partial{\xi}}{\partial \mu_0}
  &= 0\\
 \frac{\partial{\xi}}{\partial \sigma_0}
  &= 0 \\
 \frac{\partial{\xi}}{\partial \xi_0}
  &= 1
\end{align*}
}


\section{Second-order derivatives for $\lambda$:
  expressions}

\subsection*{Raw expressions}

\begin{maxima}
  d2lambdadmu_0dmu_0: diff(lambda, mu_0, 1, mu_0, 1)
\end{maxima}
\begin{maxima}
  d2lambdadmu_0dsigma_0: diff(lambda, mu_0, 1, sigma_0, 1)
\end{maxima}
\begin{maxima}
  d2lambdadmu_0dxi_0: diff(lambda, mu_0, 1, xi_0, 1)
\end{maxima}

{\color{MonVertF}
\begin{align*}
  \frac{\partial^2{\lambda}}{\partial \mu_0^2}
  &=
\begin{maxima}
  tex(d2lambdadmu_0dmu_0)
\end{maxima}\\
 \frac{\partial^2{\lambda}}{\partial \mu_0 \partial \sigma_0}
  &=
\begin{maxima}
  tex(d2lambdadmu_0dsigma_0)
\end{maxima}\\
\frac{\partial^2{\lambda}}{\partial \mu_0 \partial \xi_0}
  &=
\begin{maxima}
  tex(d2lambdadmu_0dxi_0)
\end{maxima}
\end{align*}
}

\subsection*{Simplified expressions}

{\color{red}
\begin{align*}
  \frac{\partial^2{\lambda}}{\partial \mu_0^2}
  &= -\left[1 + \xi_0\right] \, \frac{\lambda}{C^2 \sigma_0^2}\\
 \frac{\partial^2{\lambda}}{\partial \mu_0 \partial \sigma_0}
  &=  - \frac{\lambda}{C \sigma_0^2} \,
    \left\{ 1 - \left[1 + \xi_0\right] \, \frac{z_0}{C}
   \right\}\\
\frac{\partial^2{\lambda}}{\partial \mu_0 \partial \xi_0}
  &= \frac{\lambda}{C \sigma_0 \xi_0}\, \left\{ -\left[1 + \xi_0\right] \,
    \frac{z_0}{C} + \frac{\log C}{\xi_0} 
    \right\}
\end{align*}
}


\subsection*{Raw expressions}

\begin{maxima}
  d2lambdadsigma_0dmu_0: diff(lambda, sigma_0, 1, mu_0, 1)
\end{maxima}
\begin{maxima}
  d2lambdadsigma_0dsigma_0: diff(lambda, sigma_0, 1, sigma_0, 1)
\end{maxima}
\begin{maxima}
  d2lambdadsigma_0dxi_0: diff(lambda, sigma_0, 1, xi_0, 1)
\end{maxima}

{\color{MonVertF}
\begin{align*}
 %  \frac{\partial^2{\lambda}}{\partial \sigma_0 \partial \mu_0}
%   &=
% \begin{maxima}
%   tex(d2lambdadsigma_0dmu_0)
% \end{maxima}\\
 \frac{\partial^2{\lambda}}{\partial \sigma_0^2}
  &=
\begin{maxima}
  tex(d2lambdadsigma_0dsigma_0)
\end{maxima}\\
\frac{\partial^2{\lambda}}{\partial \sigma_0 \partial \xi_0}
  &=
\begin{maxima}
  tex(d2lambdadsigma_0dxi_0)
\end{maxima}
\end{align*}
}

\subsection*{Simplified expressions}

{\color{red}
\begin{align*}
 \frac{\partial^2{\lambda}}{\partial \sigma_0^2}
  &=  - \frac{\lambda z_0}{C \sigma_0^2} \,
    \left\{ 2 - \left[1 + \xi_0\right] \, \frac{z_0}{C}
    \right\} \\
\frac{\partial^2{\lambda}}{\partial \sigma_0 \partial \xi_0}
  &= \frac{\lambda z_0}{C \sigma_0 \xi_0}\, \left\{ -\left[1 + \xi_0\right] \,
    \frac{z_0}{C} + \frac{\log C}{\xi_0} 
    \right\} = z_0 \, \frac{\partial^2{\lambda}}{\partial \mu_0 \partial \xi_0}
\end{align*}
}

\section{Second-order derivatives for $\lambda$:
  limit for $\xi \to 0$}

{\color{MonVertF}
\begin{align*}
\lim_{\xi \to 0} \frac{\partial^2 \lambda}{\partial \mu_0^2} &=
\begin{maxima}
tex(limit(d2lambdadmu_0dmu_0, xi, 0))
\end{maxima} \\
  \lim_{\xi \to 0} \frac{\partial^2 \lambda}{\partial \mu_0 \partial \sigma_0} &=
\begin{maxima}
tex(limit(d2lambdadmu_0dsigma_0, xi, 0))
\end{maxima} \\
   \lim_{\xi \to 0} \frac{\partial^2 \lambda}{\partial \mu_0 \partial \xi_0} &=
\begin{maxima}
tex(limit(d2lambdadmu_0dxi_0, xi, 0))
\end{maxima} \\
   \lim_{\xi \to 0} \frac{\partial^2 \lambda}{\partial \sigma_0^2} &=
\begin{maxima}
tex(limit(d2lambdadsigma_0dsigma_0, xi, 0))
\end{maxima} \\
   \lim_{\xi \to 0} \frac{\partial^2 \lambda}{\partial \sigma_0 \partial \xi_0} &=
\begin{maxima}
tex(limit(d2lambdadmu_0dsigma_0, xi, 0))
\end{maxima}
\end{align*}}

\section{Second-order derivatives for $\sigma$:
  expressions}

{\color{red}
\begin{align*}
  \frac{\partial^2{\sigma}}{\partial \mu_0^2}
  &= 0\\
 \frac{\partial^2{\sigma}}{\partial \mu_0 \partial \sigma_0}
  &= 0\\
\frac{\partial^2{\sigma}}{\partial \mu_0 \partial \xi_0}
  &= -1
\end{align*}
}