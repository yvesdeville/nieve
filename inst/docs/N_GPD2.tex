
%%\chapter{GPD2: the Two-Parameter Generalised Pareto Distribution}

\section{The log-density $\log f$}

\subsection{Expression}

Define $z$ by
\begin{equation}
  \label{eq:defz}
  z := \frac{y}{\sigma}.
\end{equation}

\begin{maxima}%
  z: y / sigma
\end{maxima}%
\begin{maxima}%
  A: log(1 + xi * z)
\end{maxima}
\begin{maxima}%
  B: z / (1 + xi * z)
\end{maxima}%
\noindent
Define $A$  and $B$ by
\begin{equation}
  \label{eq:defAB}
A := 
\begin{maxima}%
tex(A)
\end{maxima},%
\qquad
B :=
\begin{maxima}%
tex(B)
\end{maxima}%,
\end{equation}
%% -----------------------------------------------------------------------
\begin{maxima}%
  g: -log(sigma) - (1 / xi + 1) * A 
\end{maxima}%
\noindent then
\[
\log f = 
\begin{maxima}
  tex(g)
\end{maxima}.
\]

%% -----------------------------------------------------------------------
\begin{maxima}
  g1: diff(g, sigma, 1)
\end{maxima}%
%%
\begin{maxima}
  g2: diff(g, xi, 1)
\end{maxima}%
%-----------------------------------------------------------------------


\subsection{Taylor expansion for $\xi \approx 0$}
%% ========================================================
\subsubsection*{Raw expression}
\begin{maxima}
  tg: taylor(g, xi, 0, 2)
\end{maxima}
The Taylor approximation of $\log f$ for $\xi \approx  0$ is 
{\color{MonVertF}
  $$
  \log f =
\begin{maxima}
  tex(tg)
\end{maxima}
$$
}

\subsubsection*{Simplified expression}

{\color{red}
  \begin{equation*}
    \log f = - \left[ \log \sigma + z\right] + \frac{z (z - 2) }{2} \, \xi -
    \frac{z^2 (2 z - 3)}{6} \, \xi^2 + o(\xi^2)
  \end{equation*}
}
  
\subsection{First-order derivatives: expressions}
%% =======================================================
\subsubsection*{Raw expressions}

{\color{MonVertF}
\begin{align*}
\frac{\partial \log f}{\partial \sigma}  &=
\begin{maxima}
 tex(g1)
\end{maxima}
\\
\frac{\partial \log f}{\partial \xi} &=
\begin{maxima}
 tex(g2)
\end{maxima}
\end{align*}
}

\subsubsection*{Simplified expressions}

{\color{red}
\begin{align*}
  \frac{\partial \log f}{\partial \sigma}
  &= - \frac{1}{\sigma}\, \left[1 - (\xi + 1) B \right]\\
  \frac{\partial \log f}{\partial \xi}
  &= \frac{1}{\xi^2} \, \left[A - \xi (\xi +1) B \right]
\end{align*}
}


\subsection{First-order derivatives: Taylor expansions for $\xi \approx 0$}
%% =============================================================================
\subsubsection*{Raw expressions}
%% ---------------------------------
\begin{maxima}
  tg1: taylor(g1, xi, 0, 1)
\end{maxima}
\begin{maxima}
  tg2: taylor(g2, xi, 0, 1)
\end{maxima}
For $\xi \approx  0$ we have
{\color{MonVertF}
\begin{align*}
  \frac{\partial \log f}{\partial \sigma} &=
\begin{maxima}
  tex(tg1)
\end{maxima} \\
\frac{\partial \log f}{\partial \xi} &=
\begin{maxima}
  tex(tg2)
\end{maxima}
\end{align*}
}

\subsubsection*{Simplified expressions}
%%------------------------------------------
{\color{red}
  \begin{align*}
    \frac{\partial \log f}{\partial \sigma}
    &= \frac{z - 1}{\sigma} - \frac{z (z - 1)}{\sigma} \,\xi + o(\xi) \\
    \frac{\partial \log f}{\partial \xi}
    &=  \frac{z(z - 2)}{2} - \frac{z^2(2z - 3)}{3}\, \xi + o(\xi) 
  \end{align*}
}
  
\subsection{Second-order derivatives: expressions}
% %% ============================================

\subsubsection*{Raw expressions} 
 
 \begin{maxima}
   g11: diff(g, sigma, 2)
 \end{maxima}%
 \begin{maxima}
   g11b: letsimp(g11)
 \end{maxima}%
 
 \begin{maxima}
   g12: diff(g, sigma, 1, xi, 1)
 \end{maxima}%
 
 \begin{maxima}
   g12b: letsimp(g12)
 \end{maxima}%
 
 \begin{maxima}
   g22: diff(g, xi, 2)
 \end{maxima}%
 
 \begin{maxima}
   g22b: letsimp(g22)
 \end{maxima}%
 
{\color{MonVertF}  
$$
\frac{\partial^2}{\partial \sigma^2} \,\log f = 
\begin{maxima}
  tex(g11b)
\end{maxima}
$$

$$
\frac{\partial^2}{\partial \sigma \partial \xi} \, \log f =
\begin{maxima}
  tex(g12b)
\end{maxima}
$$

$$
 \frac{\partial^2}{\partial \xi^2}\, \log f =
\begin{maxima}
 tex(g22b)
\end{maxima}
$$
}

\subsubsection*{Simplified  expressions}

{\color{red}
\begin{align*}
  \frac{\partial^2}{\partial \sigma^2} \,\log f
  &= \frac{1}{\sigma^2}\, \left[ 1 - 2 (\xi + 1) B + \xi (\xi + 1) B^2\right] \\
 \frac{\partial^2}{\partial \sigma \partial \xi} \,\log f 
  &= \frac{1}{\sigma}\, \left[ B - (\xi + 1) B^2\right] \\
    \frac{\partial^2}{\partial \xi^2} \,\log f 
  &= \frac{1}{\xi^3}\, \left[-2 A + 2 \xi  B + \xi^2 (\xi + 1) B^2\right]
\end{align*}
}



\subsection{Second-order derivatives: limits for $\xi \to 0$}
%% =====================================================================

\subsubsection*{Raw expressions} 

Here are the limits for $\xi \to 0$

{\color{MonVertF}
\begin{align*}
\lim_{\xi \to 0} \frac{\partial^2 \log f}{\partial \sigma^2} &=
\begin{maxima}
tex(limit (g11, xi, 0))
\end{maxima} \\
\lim_{\xi \to 0} \frac{\partial^2 \log f}{\partial \sigma \partial \xi} &=
\begin{maxima}
tex(limit (g12, xi, 0))
\end{maxima} \\
\lim_{\xi \to 0} \frac{\partial^2 \log f}{\partial \xi^2} &=
\begin{maxima}
tex(limit (g22, xi, 0))
\end{maxima}
\end{align*}}

\subsubsection*{Simplified expressions} 

The corresponding simplified expressions are
{\color{red}
\begin{align*}
\lim_{\xi \to 0} \frac{\partial^2 \log f}{\partial \sigma^2} &=
\frac{1-2z}{\sigma^2}\\
\lim_{\xi \to 0} \frac{\partial^2 \log f}{\partial \sigma \partial \xi} &=
- \frac{z (z - 1)}{\sigma}\\
\lim_{\xi \to 0} \frac{\partial^2 \log f}{\partial \xi^2} &= 
- \frac{z^2 (2 z - 3)}{3}.
\end{align*}
}

%%========================================================
\section{Cumulated hazard (log-survival) $H = \log S$}

\subsection{Expression}

The cumulated hazard $H(y) = \log S(y)$ 
\begin{maxima}
  H: A / xi
\end{maxima}

$$
H :=
\begin{maxima}
  tex(H)
\end{maxima}
$$

\subsection{Taylor approximation for $\xi \approx 0$}

\subsubsection*{Raw expression}
\begin{maxima}
  tH: taylor(H, xi, 0, 2)
\end{maxima}

{\color{MonVertF}
\begin{equation*}
  H = 
  \begin{maxima}
    tex(tH)
  \end{maxima}
\end{equation*}
}

\subsubsection*{Simplified expression}
{\color{red}
  \begin{equation}
    \label{eq:TH}
    H = z - \frac{z^2}{2}\, \xi +  \frac{z^3}{3}\, \xi^2 + o (\xi^2)
  \end{equation}
}
%
\begin{maxima}
  H1: diff(H, sigma, 1)
\end{maxima}%
%%
\begin{maxima}
  H2: diff(H, xi, 1)
\end{maxima}%


\subsection{First-order derivatives: expressions}
%-----------------------------------------------------------------------

\subsubsection*{Raw expressions}

{\color{MonVertF}
\begin{align*}
\frac{\partial H}{\partial \sigma}  &=
\begin{maxima}
 tex(H1)
\end{maxima}
\\
\frac{\partial H}{\partial \xi} &=
\begin{maxima}
 tex(H2)
\end{maxima}
\end{align*}
}

\subsubsection*{Simplified expressions}

{\color{red}
\begin{align*}
  \frac{\partial H}{\partial \sigma}
  &= - \frac{1}{\sigma}\, B  \\
  \frac{\partial H}{\partial \xi}
  &= -\frac{1}{\xi^2} \, \left[A - \xi B \right]
\end{align*}
}


\subsection{First-order derivatives: Taylor expansions for $\xi \approx 0$}
%% ========================================================
\subsubsection*{Raw expressions}
\begin{maxima}
  tH1: taylor(H1, xi, 0, 1)
\end{maxima}
\begin{maxima}
  tH2: taylor(H2, xi, 0, 1)
\end{maxima}

Here are the limits for $\xi \approx  0$
{\color{MonVertF}
\begin{align*}
  \frac{\partial H}{\partial \sigma} &=
\begin{maxima}
  tex(tH1)
\end{maxima}\\
  \frac{\partial H}{\partial \xi} &=
\begin{maxima}
  tex(tH2)
\end{maxima}
\end{align*}
}

\subsubsection*{Simplified expressions}

{\color{red}
  \begin{align*}
    \frac{\partial H}{\partial \sigma}
    &=  - \frac{z}{\sigma} + \frac{z^2}{\sigma}\, \xi+ o(\xi) \\
    \frac{\partial H}{\partial \xi}
    &=  - \frac{z^2}{2} + \frac{2 z^3}{3} \, \xi  + o(\xi)
  \end{align*}
}
  
\subsection{Second-order derivatives: expressions}
% %% ============================================
 \subsubsection*{Raw expressions}
 
 
 \begin{maxima}
   H11: diff(H, sigma, 2)
 \end{maxima}%
 \begin{maxima}
   H11b: letsimp(H11)
 \end{maxima}%
 
 \begin{maxima}
   H12: diff(H, sigma, 1, xi, 1)
 \end{maxima}%
 
 \begin{maxima}
   H12b: letsimp(H12)
 \end{maxima}%
 
 \begin{maxima}
   H22: diff(H, xi, 2)
 \end{maxima}%
 
 \begin{maxima}
   H22b: letsimp(H22)
 \end{maxima}%
 
{\color{MonVertF}  
\begin{align*}
\frac{\partial^2}{\partial \sigma^2} \, H &= 
\begin{maxima}
  tex(H11b)
\end{maxima}\\
\frac{\partial^2}{\partial \sigma \partial \xi} \, H &=
\begin{maxima}
  tex(H12b)
\end{maxima}\\
 \frac{\partial^2}{\partial \xi^2}\, H &=
\begin{maxima}
 tex(H22b)
\end{maxima}
\end{align*}                                         
}

\subsubsection*{Simplified  expressions}

{\color{red}
\begin{align*}
  \frac{\partial^2}{\partial \sigma^2} \,H
  &=  \frac{1}{\sigma^2}\, \left[2 B - \xi B^2 \right]\\
 \frac{\partial^2}{\partial \sigma \partial \xi} \, H 
  &=  \frac{1}{\sigma}\, B^2\\
    \frac{\partial^2}{\partial \xi^2} \, H 
  &= \frac{1}{\xi^3}\,\left[2 A - 2 \xi B - \xi^2 B ^2 \right]
\end{align*}
}



\subsection{Second-order derivatives: limits for $\xi \to 0$}

%%========================================================
Here are the limits for $\xi \to 0$

{\color{MonVertF}
\begin{align*}
\lim_{\xi \to 0} \frac{\partial^2 H}{\partial \sigma^2} &=
\begin{maxima}
tex(limit (H11, xi, 0))
\end{maxima} \\
\lim_{\xi \to 0} \frac{\partial^2 H}{\partial \sigma \partial \xi} &=
\begin{maxima}
tex(limit (H12, xi, 0))
\end{maxima} \\
\lim_{\xi \to 0} \frac{\partial^2 H}{\partial \xi^2} &=
\begin{maxima}
tex(limit (H22, xi, 0))
\end{maxima}
\end{align*}}
The corresponding simplified expresions are
{\color{red}
\begin{align*}
  \lim_{\xi \to 0} \frac{\partial^2 H}{\partial \sigma^2}
  &= \frac{2z}{\sigma^2}\\
  \lim_{\xi \to 0} \frac{\partial^2 H}{\partial \sigma \partial \xi}
  &= \frac{z^2}{\sigma}\\
  \lim_{\xi \to 0} \frac{\partial^2 H}{\partial \xi^2}
    &= \frac{2 z^3}{3}.
\end{align*}
}






%%=================================================================================
\section{Quantile or return period}
%%
\subsection{Expression}
% -------------------------
The quantile corresponding to an exceedance probability $q := 1 -p$ for $\xi \neq 0$
\begin{maxima}
  rho: sigma * (q^(-xi) - 1) / xi
\end{maxima}
%%
\[
\rho=
\begin{maxima}
 tex(rho)
\end{maxima}
\]


\subsection{Taylor approximation for $\xi \approx 0$}
% --------------------------------------------------------------
\begin{maxima}
 trho: taylor(rho, xi, 0, 2)
\end{maxima}

{\color{MonVertF}
  \begin{equation}
    \label{eq:Trho}
    \begin{maxima}
      tex(trho)
    \end{maxima}
  \end{equation}
}

%%
\begin{maxima}
  rho1: diff(rho, sigma, 1)
\end{maxima}
%%
\begin{maxima}
  rho2: diff(rho, xi, 1)
\end{maxima}


\subsection{First-order derivatives: expressions}
%%========================================================
When $\xi \neq 0$ we have
\begin{align*}
\frac{\partial \rho}{\partial \sigma}  &=
\begin{maxima}
 tex(rho1)
\end{maxima}
\\
\frac{\partial \rho}{\partial \xi} &=
\begin{maxima}
 tex(rho2)
\end{maxima}
\end{align*}


\subsection{First-order derivatives: Taylor approximation}
%%========================================================

\begin{maxima}
  trho1: taylor(rho1, xi, 0, 1)
\end{maxima}

\begin{maxima}
  trho2: taylor(rho2, xi, 0, 1)
\end{maxima}

{\color{MonVertF}
\begin{align*}
\frac{\partial \rho}{\partial \sigma}  &=
\begin{maxima}
 tex(trho1)
\end{maxima}
\\
\frac{\partial \rho}{\partial \xi} &=
\begin{maxima}
 tex(trho2)
\end{maxima}
\end{align*}
}

% {\color{MonVertF}
% \begin{align*}
% \lim_{\xi \to 0} \frac{\partial \rho}{\partial \sigma} &= 
% \begin{maxima}
%  tex(limit(rho1, xi, 0))
% \end{maxima}
% \\
% \lim_{\xi \to 0} \frac{\partial \rho}{\partial \xi} &= 
% \begin{maxima}
%  tex(limit(rho2, xi, 0))
% \end{maxima}
% \end{align*}}

\subsection{Second-order derivatives: expressions}
%% ========================================================

\subsubsection*{Raw expressions}
\begin{maxima}
  rho11:  diff(rho, sigma, 2)
\end{maxima}

\begin{maxima}
  rho12:  diff(rho, sigma, 1, xi, 1)
\end{maxima}

\begin{maxima}
  rho22:  diff(rho, xi, 2)
\end{maxima}

{\color{MonVertF}
\begin{align*}
\frac{\partial^2 \rho}{\partial \sigma^2} &= 
\begin{maxima}
  tex(rho11)
\end{maxima} \\ 
\frac{\partial^2 \rho}{\partial \sigma \partial \xi} &= 
\begin{maxima}
  tex(rho12)
\end{maxima} \\ 
\frac{\partial^2 \rho}{\partial \xi^2} &= 
\begin{maxima}
  tex(rho22)
\end{maxima} 
\end{align*}
}


\subsubsection*{Simplified expressions}

Define $V:= [q^{-\xi} - 1] / \xi$ so that $\rho = \sigma V$, and
$$
W := \frac{\partial V}{\partial \xi}
=  - \frac{1}{\xi}\, \left\{ V + q^{-\xi} \log q \right\}.
$$



{\color{red}
\begin{align*}
  \frac{\partial^2 \rho}{\partial \sigma^2}
  &=  0 \\ 
  \frac{\partial^2 \rho}{\partial \sigma \partial \xi}
  &= W\\ 
  \frac{\partial^2 \rho}{\partial \xi^2}
  &= \frac{\sigma}{\xi^2}\, \left\{ 2 V + q^{-\xi} \log q \left[2 + \xi \log q \right] \right\}         
\end{align*}
}

\subsection{Second-order derivatives: limit for $\xi \to 0$}
%% ========================================================

{\color{MonVertF}
\begin{align*}
  \lim_{\xi \to 0} \frac{\partial^2 \rho}{\partial \sigma^2}
  &= 
\begin{maxima}
 tex(limit(rho11, xi, 0))
\end{maxima} \\
  \lim_{\xi \to 0} \frac{\partial^2 \rho}{\partial \sigma \partial\xi}
  &= 
\begin{maxima}
 tex(limit(rho12, xi, 0))
\end{maxima} \\
  \lim_{\xi \to 0} \frac{\partial^2 \rho}{\partial \xi^2}
  &= 
\begin{maxima}
 tex(limit(rho22, xi, 0))
\end{maxima}
\end{align*}
}
