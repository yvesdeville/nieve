
%% \chapter{NHPP to Poisson-GP transformations}

\section{Poisson-GP to PP}
%% ==============================

\subsection{Expression}
%% -----------------------

Let $\boldsymbol{\theta} = [\lambda,\, \sigma,\, \xi]$ and
$\boldsymbol{\theta}_0 = [\mu_0,\, \sigma_0,\, \xi_0]$ denote the
Poisson-GP and PP parameter vectors. The threshold $u$ being
considered as fixed, the relation between the PP parameters
$\boldsymbol{\theta}_0 $ and the
Poisson-GP parameters~$\boldsymbol{\theta}$ is given by
\begin{align*}
  \mu_0 &= u + \frac{(\lambda w)^\xi - 1}{\xi} \, \sigma,\\
  \sigma_0 &= (\lambda w)^\xi \sigma, \\
  \xi_0 &=  \xi.
\end{align*}

\begin{maxima}
  mu_0: u +  sigma * ((lambda * w)^xi - 1) / xi
\end{maxima}
\begin{maxima}
  sigma_0: sigma * (lambda * w)^xi
\end{maxima}
\begin{maxima}
  xi_0: xi
\end{maxima}


%\ begin{align*}
% \mu_0
% &=
% \begin{maxima}
%   tex(mu_0)
% \end{maxima}\\
% \sigma_0 &=
% \begin{maxima}
%   tex(sigma_0)
% \end{maxima}\\
% \xi_0
% &=
% \begin{maxima}
%   tex(xi_0)
% \end{maxima}
% \end{align*}

{\footnotesize
\begin{maxima}
  tmu_0: taylor(mu_0, xi, 0, 2)
\end{maxima}
\begin{maxima}
  tsigma_0: taylor(sigma_0, xi, 0, 2)
\end{maxima}
\begin{maxima}
  txi_0: taylor(xi_0, xi, 0, 2)
\end{maxima}
}

\subsection{Taylor approximation for $\xi \approx 0$}

\subsubsection*{Raw expressions}

%% \begin{landscape}
{\color{MonVertF}
\begin{align*}
\mu_0 &= 
\begin{maxima}
  tex(tmu_0)
\end{maxima} \\
\sigma_0 &= 
\begin{maxima}
  tex(tsigma_0)
\end{maxima}\\
  \xi_0 &= 
\begin{maxima}
  tex(txi_0)
\end{maxima}
\end{align*}
}
%% \end{landscape}

\subsubsection*{Simplified expressions}
With $L:= \log(\lambda w)$, we have
{\color{red}
\begin{align*}
  \mu_0 &= u + \sigma \, \left\{ L + \frac{L^2}{2} \,\xi + \frac{L^3}{3}\xi^2 \right\} + o(\xi^2) \\
  \sigma_0 &= \sigma \,  \left\{ 1 + L \,\xi + \frac{L^2}{2}\xi^2 \right\} + o(\xi^2) \\
  \xi_0 &= \xi + o(\xi^2)
\end{align*}
}


\section{First-order derivatives for $\mu_0$: expressions}

\subsection*{Raw expressions}

\begin{maxima}
  dmu_0dlambda: diff(mu_0, lambda, 1)
\end{maxima}
\begin{maxima}
  dmu_0dsigma: diff(mu_0, sigma, 1)
\end{maxima}
\begin{maxima}
  dmu_0dxi: diff(mu_0, xi, 1)
\end{maxima}

{\color{MonVertF}
\begin{align*}
  \frac{\partial \mu_0}{\partial \lambda}
  &=
\begin{maxima}
  tex(dmu_0dlambda)
\end{maxima}\\
 \frac{\partial \mu_0}{\partial \sigma}
  &=
\begin{maxima}
  tex(dmu_0dsigma)
\end{maxima}\\
\frac{\partial \mu_0 }{\partial \xi}
  &=
\begin{maxima}
  tex(dmu_0dxi)
\end{maxima}
\end{align*}
}

\subsection*{Simplified expressions}
With $\nu := \lambda w$, we have

{\color{red}
\begin{align*}
  \frac{\partial \mu_0}{\partial \lambda}
  &= \sigma \, \frac{\nu^\xi}{\lambda}\\
 \frac{\partial \mu_0}{\partial \sigma}
  &= \frac{\nu^\xi-1}{\xi}\\
\frac{\partial \mu_0 }{\partial \xi}
  &= \frac{\sigma}{\xi} \, \left\{\nu^\xi L - \dfrac{\nu^\xi - 1}{\xi} \right\} =
  \frac{\sigma}{\xi} \, \left\{\nu^\xi L - \frac{\partial \mu_0}{\partial \sigma} \right\} \\
\end{align*}
}
\section{First-order derivatives for $\mu_0$: Taylor approximation}

\subsection*{Raw expressions}
\begin{maxima}
  tdmu_0dlambda: taylor(dmu_0dlambda, xi, 0, 1)
\end{maxima}
\begin{maxima}
  tdmu_0dsigma: taylor(dmu_0dsigma, xi, 0, 1)
\end{maxima}
\begin{maxima}
  tdmu_0dxi: taylor(dmu_0dxi, xi, 0, 1)
\end{maxima}

{\color{MonVertF}
\begin{align*}
  \frac{\partial \mu_0}{\partial \lambda}
  &=
\begin{maxima}
  tex(tdmu_0dlambda)
\end{maxima}\\
 \frac{\partial \mu_0}{\partial \sigma}
  &=
\begin{maxima}
  tex(tdmu_0dsigma)
\end{maxima}\\
\frac{\partial \mu_0 }{\partial \xi}
  &=
\begin{maxima}
  tex(tdmu_0dxi)
\end{maxima}
\end{align*}
}

\subsection*{Simplified expressions}
{\color{red}
\begin{align*}
  \frac{\partial \mu_0}{\partial \lambda}
  &= \frac{\sigma}{\lambda}\, \left\{ 1 + L \, \xi \right\} + o(\xi) \\
 \frac{\partial \mu_0}{\partial \sigma}
  &= \frac{L}{2} \, \left\{ 2 + L\, \xi \right\} + o(\xi) \\
\frac{\partial \mu_0 }{\partial \xi}
  &=  \sigma \, \frac{L^2}{6} \left\{ 3  + 2 L\, \xi \right\} + o(\xi) 
\end{align*}
}

\section{First-order derivatives for $\sigma_0$: expressions}

\subsection*{Raw expressions}

\begin{maxima}
  dsigma_0dlambda: diff(sigma_0, lambda, 1)
\end{maxima}
\begin{maxima}
  dsigma_0dsigma: diff(sigma_0, sigma, 1)
\end{maxima}
\begin{maxima}
  dsigma_0dxi: diff(sigma_0, xi, 1)
\end{maxima}

{\color{MonVertF}
\begin{align*}
  \frac{\partial \sigma_0}{\partial \lambda}
  &=
    \begin{maxima}
      tex(dsigma_0dlambda)
    \end{maxima} \\
 \frac{\partial \sigma_0}{\partial \sigma}
  &=
    \begin{maxima}
      tex(dsigma_0dsigma)
    \end{maxima} \\
  \frac{\partial \sigma_0 }{\partial \xi}
  &= \begin{maxima}
      tex(dsigma_0dxi)
    \end{maxima} \\
\end{align*}
}

\subsection*{Simplified expressions}

{\color{red}
\begin{align*}
  \frac{\partial \sigma_0}{\partial \lambda}
  &= \sigma \, \xi \, \frac{\nu^\xi}{\lambda} = \xi \frac{\partial \mu_0}{\partial \lambda}\\
 \frac{\partial \sigma_0}{\partial \sigma}
  &= \nu^\xi\\
  \frac{\partial \sigma_0 }{\partial \xi}
  &= \sigma \nu^\xi L 
\end{align*}
}

\section{First-order derivatives for $\xi_0$: expressions}

{\color{red}
\begin{align*}
  \frac{\partial \xi_0}{\partial \lambda}
  &= 0\\
 \frac{\partial \xi_0}{\partial \sigma}
  &= 0\\
  \frac{\partial \xi_0 }{\partial \xi}
  &= 1
\end{align*}
}

\section{First-order derivatives for $\sigma_0$: Taylor approximation}

\subsection*{Raw expressions}
\begin{maxima}
  tdsigma_0dlambda: taylor(dsigma_0dlambda, xi, 0, 1)
\end{maxima}
\begin{maxima}
  tdsigma_0dsigma: taylor(dsigma_0dsigma, xi, 0, 1)
\end{maxima}
\begin{maxima}
  tdsigma_0dxi: taylor(dsigma_0dxi, xi, 0, 1)
\end{maxima}

{\color{MonVertF}
\begin{align*}
  \frac{\partial \sigma_0}{\partial \lambda}
  &=
\begin{maxima}
  tex(tdsigma_0dlambda)
\end{maxima}\\
 \frac{\partial \sigma_0}{\partial \sigma}
  &=
\begin{maxima}
  tex(tdsigma_0dsigma)
\end{maxima}\\
\frac{\partial \sigma_0 }{\partial \xi}
  &=
\begin{maxima}
  tex(tdsigma_0dxi)
\end{maxima}
\end{align*}
}

\subsection*{Simplified expressions}

{\color{red}
\begin{align*}
  \frac{\partial \sigma_0}{\partial \lambda}
  &= \frac{\sigma}{\lambda} \, \xi + o(\xi)\\
 \frac{\partial \sigma_0}{\partial \sigma}
  &= 1 + L \xi + o(\xi)\\
\frac{\partial \sigma_0 }{\partial \xi}
  &= \sigma L \, \left\{ 1 + L \xi \right\} +  o(\xi)
\end{align*}
}

