\documentclass[11pt]{article}\usepackage[]{graphicx}\usepackage[]{xcolor}
% maxwidth is the original width if it is less than linewidth
% otherwise use linewidth (to make sure the graphics do not exceed the margin)
\makeatletter
\def\maxwidth{ %
  \ifdim\Gin@nat@width>\linewidth
    \linewidth
  \else
    \Gin@nat@width
  \fi
}
\makeatother

\definecolor{fgcolor}{rgb}{0.345, 0.345, 0.345}
\newcommand{\hlnum}[1]{\textcolor[rgb]{0.686,0.059,0.569}{#1}}%
\newcommand{\hlstr}[1]{\textcolor[rgb]{0.192,0.494,0.8}{#1}}%
\newcommand{\hlcom}[1]{\textcolor[rgb]{0.678,0.584,0.686}{\textit{#1}}}%
\newcommand{\hlopt}[1]{\textcolor[rgb]{0,0,0}{#1}}%
\newcommand{\hlstd}[1]{\textcolor[rgb]{0.345,0.345,0.345}{#1}}%
\newcommand{\hlkwa}[1]{\textcolor[rgb]{0.161,0.373,0.58}{\textbf{#1}}}%
\newcommand{\hlkwb}[1]{\textcolor[rgb]{0.69,0.353,0.396}{#1}}%
\newcommand{\hlkwc}[1]{\textcolor[rgb]{0.333,0.667,0.333}{#1}}%
\newcommand{\hlkwd}[1]{\textcolor[rgb]{0.737,0.353,0.396}{\textbf{#1}}}%
\let\hlipl\hlkwb

\usepackage{framed}
\makeatletter
\newenvironment{kframe}{%
 \def\at@end@of@kframe{}%
 \ifinner\ifhmode%
  \def\at@end@of@kframe{\end{minipage}}%
  \begin{minipage}{\columnwidth}%
 \fi\fi%
 \def\FrameCommand##1{\hskip\@totalleftmargin \hskip-\fboxsep
 \colorbox{shadecolor}{##1}\hskip-\fboxsep
     % There is no \\@totalrightmargin, so:
     \hskip-\linewidth \hskip-\@totalleftmargin \hskip\columnwidth}%
 \MakeFramed {\advance\hsize-\width
   \@totalleftmargin\z@ \linewidth\hsize
   \@setminipage}}%
 {\par\unskip\endMakeFramed%
 \at@end@of@kframe}
\makeatother

\definecolor{shadecolor}{rgb}{.97, .97, .97}
\definecolor{messagecolor}{rgb}{0, 0, 0}
\definecolor{warningcolor}{rgb}{1, 0, 1}
\definecolor{errorcolor}{rgb}{1, 0, 0}
\newenvironment{knitrout}{}{} % an empty environment to be redefined in TeX

\usepackage{alltt}

%% ===========================================================================
%% Copyright Yves Deville <deville.yves@alpestat.com> 2022
%% ===========================================================================

\usepackage{amsmath,amssymb,amsthm}
\usepackage[english]{babel}
\usepackage[amsmath]{maxiplot}
\usepackage{fullpage}
\usepackage{lscape}
\usepackage{hyperref}
\usepackage{tikz}
\definecolor{MonVert}{rgb}{0.398,0.801,0.000} 
\definecolor{MonVertF}{rgb}{0.13,0.54,0.13}
\definecolor{MonRouge}{rgb}{0.600,0.060,0.360} 
\definecolor{MonBleu}{rgb}{0.000,0.602,0.801} 
\definecolor{SteelBlue2}{rgb}{0.359375,0.671875,0.9296875}
\definecolor{orange}{rgb}{1.0,0.6470,0.0}
\definecolor{SteelBlue4}{rgb}{0.212, 0.392, 0.545}
\definecolor{MonJaune}{rgb}{0.996,0.996,0.875}
\definecolor{orange1}{rgb}{0.996,0.645,0}
\definecolor{PaleVioletRed}{rgb}{0.855,0.438,0.574}
\newcommand{\m}{\mathbf}   
\newcommand{\bs}{\boldsymbol}
\newcommand{\pkg}[1]{\textbf{#1}}
\newcommand{\code}[1]{\texttt{#1}}
\newcommand{\Gre}[1]{{\color{MonVertF}#1}}

\title{\bf \Large The \textbf{yaev} package: Yet Another Extreme Value package?}

\author{Yves Deville \href{mailto:deville.yves@alpestat.com}%%
  {deville.yves@alpestat.com} }
\IfFileExists{upquote.sty}{\usepackage{upquote}}{}
\begin{document}
\maketitle{}
\tableofcontents{}

\section{Probability functions of Extreme-Value distributions}
  
The probability functions for the GPD and GEV distributions depend
smoothly on the parameters: they are infinitely differentiable
functions of the parameters. However these functions are not analytic
functions of the parameters and a singularity exists for all the
functions when the shape parameter say $\xi$ is zero.  In practice,
the functions are given with different formulas depending on whether
$\xi$ is zero or not; the formulas for $\xi = 0$ relate to the
exponential and Gumbel distributions and correspond to the limit for
$\xi \to 0$ of the functions given by the formulas for $\xi \neq 0$.
As an example consider the quantile function of the Generalized Pareto
distribution with shape $\xi$ and unit scale
\begin{equation}
\label{eq:Quant}
q(p) = \begin{cases}
  [(1 - p)^{-\xi} - 1] / \xi & \xi \neq 0 \\
  -\log(1 - p) & \xi = 0,
\end{cases} \qquad 0 < p < 1.
\end{equation}
It can be shown that for $\xi \approx 0$ whatever be $p$
$$
q \approx - \log(1-p), \qquad
\frac{\partial q}{\partial \xi} \approx \frac{1}{2}\, \log^2(1-p), \qquad
\frac{\partial^2 q}{\partial \xi^2} \approx -\frac{1}{3}\, \log^2(1-p).
$$
It is quite easy to obtain expressions for the the derivatives
w.r.t. $\xi$ using the definition (\ref{eq:Quant}). We can even rely
on the symbolic differentiation method \verb@D@ available in R which,
as opposed to me and many other humans, never makes any mistake when
differentiating.

\begin{knitrout}\footnotesize
\definecolor{shadecolor}{rgb}{0.969, 0.969, 0.969}\color{fgcolor}\begin{kframe}
\begin{alltt}
\hlstd{qEx} \hlkwb{<-} \hlkwa{function}\hlstd{(}\hlkwc{p}\hlstd{,} \hlkwc{xi}\hlstd{) ((}\hlnum{1} \hlopt{-} \hlstd{p)}\hlopt{^}\hlstd{(}\hlopt{-}\hlstd{xi)} \hlopt{-} \hlnum{1}\hlstd{)} \hlopt{/} \hlstd{xi}
\hlstd{dqEx} \hlkwb{<-} \hlkwd{D}\hlstd{(}\hlkwd{expression}\hlstd{(((}\hlnum{1} \hlopt{-} \hlstd{p)}\hlopt{^}\hlstd{(}\hlopt{-}\hlstd{xi)} \hlopt{-} \hlnum{1}\hlstd{)} \hlopt{/} \hlstd{xi),} \hlkwc{name} \hlstd{=} \hlstr{"xi"}\hlstd{)}
\hlstd{d2qEx} \hlkwb{<-} \hlkwd{D}\hlstd{(dqEx,} \hlkwc{name} \hlstd{=} \hlstr{"xi"}\hlstd{)}
\hlstd{p} \hlkwb{<-} \hlnum{0.99}\hlstd{; q} \hlkwb{<-} \hlnum{1} \hlopt{-} \hlstd{p}
\hlkwa{for} \hlstd{(xi} \hlkwa{in} \hlkwd{c}\hlstd{(}\hlnum{1e-4}\hlstd{,} \hlnum{1e-6}\hlstd{,} \hlnum{1e-8}\hlstd{)) \{}
    \hlstd{r} \hlkwb{<-} \hlkwd{rbind}\hlstd{(}\hlstr{"ord 0"} \hlstd{=} \hlkwd{c}\hlstd{(}\hlstr{"lim"} \hlstd{=} \hlopt{-}\hlkwd{log}\hlstd{(q),} \hlstr{"der"} \hlstd{=} \hlkwd{qEx}\hlstd{(}\hlkwc{p} \hlstd{= p,} \hlkwc{xi} \hlstd{= xi)),}
               \hlstr{"ord 1"} \hlstd{=} \hlkwd{c}\hlstd{(}\hlstr{"lim"} \hlstd{=} \hlkwd{log}\hlstd{(q)}\hlopt{^}\hlnum{2} \hlopt{/} \hlnum{2}\hlstd{,} \hlstr{"der"} \hlstd{=} \hlkwd{eval}\hlstd{(dqEx,} \hlkwd{list}\hlstd{(}\hlkwc{p} \hlstd{= p,} \hlkwc{xi} \hlstd{= xi))),}
               \hlstr{"ord 2"} \hlstd{=} \hlkwd{c}\hlstd{(}\hlstr{"lim"} \hlstd{=} \hlopt{-}\hlkwd{log}\hlstd{(q)}\hlopt{^}\hlnum{3} \hlopt{/} \hlnum{3}\hlstd{,} \hlstr{"der"} \hlstd{=} \hlkwd{eval}\hlstd{(d2qEx,} \hlkwd{list}\hlstd{(}\hlkwc{x} \hlstd{= p,} \hlkwc{xi} \hlstd{= xi))))}
    \hlkwd{cat}\hlstd{(}\hlstr{"xi = "}\hlstd{, xi,} \hlstr{"\textbackslash{}n"}\hlstd{)}
    \hlkwd{print}\hlstd{(r)}
\hlstd{\}}
\end{alltt}
\begin{verbatim}
## xi =  1e-04 
##            lim       der
## ord 0  4.60517  4.606231
## ord 1 10.60380 10.607052
## ord 2 32.55486 32.566137
## xi =  1e-06 
##            lim       der
## ord 0  4.60517  4.605181
## ord 1 10.60380 10.603811
## ord 2 32.55486 68.213867
## xi =  1e-08 
##            lim           der
## ord 0  4.60517  4.605170e+00
## ord 1 10.60380  1.085129e+01
## ord 2 32.55486 -4.949912e+07
\end{verbatim}
\end{kframe}
\end{knitrout}

\noindent
We see that the formula for the function works fine. However, the
formula for the 2-nd order derivative can be completely wrong when
$\xi$ is about $\text{1e-6}$ and the formula for the 1-st order
derivative can also be wrong when $\xi$ is about $\text{1e-8}$. The
reason is that the formulas for the derivatives involve difference
and/or fractions or small quantities since $\xi$ or $\xi^2$ comes at
the denominator. As a general rule the derivatives with higher order
are more difficult to evaluate numerically, since they involve more
complex expressions.  Note that using a shape $\xi$ with
$|\xi| \leqslant \text{1e-6}$ is quite common in EVA because the
values of $\xi$ used in practice are often quite small, and moreover 
very small values of $\xi$ are often used in the initialisation of the
Maximum-Likelihood (ML) optimization.

Although not yet widespread, the use of the exact formulas for the
derivatives w.r.t. the parameters can be of great help in the
optimization tasks required in EVA. These tasks of course involve the
ML estimation but also profile-likelihood inference for models with
covariates. Differential equations methods can be also used to derive
confidence intervals. Note that the use of formulas for the
derivatives is called \textit{symbolic} differentiation and  differs
from \textit{automatic} differentiation as increasingly available.

Our strategy consists in fixing a small $\epsilon >0$ and use the
formulas for $\xi \neq 0$ only when $|\xi| > \epsilon$. When
$|\xi| \leqslant \epsilon$, we use a few terms from the Taylor series
at $\xi =0$ e.g.,
$$
q(p;\,\xi) \approx q(p;\, 0) +
\left. \partial_{\xi} q(p;\, \xi)\right|_{\xi = 0} \times \xi
+ \frac{1}{2} \,
\left.\partial^2_{\xi,\xi} q(p;\, \xi)\right|_{\xi = 0}
\times \xi^2 + o(\xi^2).
$$
In order to maintain the consistency between the derivatives, it seems
good to use the same $\epsilon$ for all the derivatives and use
consistent Taylor approximations, so use the order $1$ for the
derivative~$\partial_{\xi} q$ and the order order~$0$
for~$\partial^2_{\xi,\xi} q$ or for a crossed derivative such
as~$\partial^2_{\sigma,\xi} q$. Since the $2$-nd order derivatives can
be required, we must take a value for $\epsilon$ which is not too
small: $\text{1e-4}$ or $\text{1e-5}$, not much smaller. Note
that~$\epsilon$ gives the level of error for the $2$-nd order
dervivative; since the error on the function is~$O(\xi^3)$, using
$\epsilon = \text{1e-4}$ leads to an error of order~$\text{1e-12}$ on
the function, which seems acceptable in practice. This kind of
approximation is used in some codes of the \textbf{revdbayes} package
by Paul Northrop, see the code if the \code{dgev} and \code{pgev}
functions on
\href{https://github.com/paulnorthrop/revdbayes/blob/master/R/distributions.R}{GitHub
  repos}.

\section{Deriving the formulas}

The reports provided with \textbf{yaev}: 
\href{https://github.com/yvesdeville/yaev/blob/main/inst/doc/GEV.pdf}{GEV.pdf},
\href{https://github.com/yvesdeville/yaev/blob/main/inst/doc/GPD2.pdf}{GPD2.pdf},
\href{https://github.com/yvesdeville/yaev/blob/main/inst/doc/PoisGP2PP.pdf}{PoisGP2PP.pdf} and
\href{https://github.com/yvesdeville/yaev/blob/main/inst/doc/PP2PoisGP.pdf}{PP2PoisGP.pdf}
  give the exact expressions for
the first-order and the second-order derivatives of the probability
functions w.r.t. the parameters and also provides workable
approximations for the case $\xi \approx 0$. We used the
\href{https://maxima.sourceforge.io/}{Maxima Computer Algebra System}
along with the
\href{https://maxima.sourceforge.io/contrib/maxiplot/maxiplot.sty}{maxiplot}
package for \LaTeX{}.

\begin{itemize}
\item The {\color{MonVertF} \bf raw expressions given by Maxima are
    reported in green}. The expressions can be regarded as exact, not
  being influenced by manual computations. However these formulas are
  usually difficult to use in a compiled code and require some manual
  transformation for this aim.

\item The {\color{red} \bf simplified expressions are reported in
    red}.  These expressions are derived by us from the raw
  expressions; they are influenced by manual computations hence could
  in principle contain errors, although they have been carefully
  checked. These formulas are used to write the compiled code.  They
  often use auxiliary variables that are shared across several
  formulas.
  
\end{itemize}

\section{Testing the derivatives}

The \textbf{yaev} package comes with a series of tests in the format
of the \textbf{testthat} package. The \textbf{numDeriv} package is
used to compute the derivatives by numeric differentiation up to the
order~2; these derivatives are compared to those provided by the
formulas.

A quite difficult task when checking derivatives is to give a
threshold used to decide if the difference between the numeric
derivative and the symbolic derivative, say the ``error'', is
acceptable or not. This error has two sources: one is the numerical
evaluation of the symbolic derivative or of its approximation for
$\xi \approx 0$ (see example above), and the other is the
approximation used in numeric differentiation where the limit defining
the derivative is replaced by a finite difference.  We should use
small values of $\xi$ with $|\xi| < \epsilon$ to check that the
approximation for small~$\xi$ is correct, although we can only test
the approximation at the first order by doing so. Then, with a good
choice of $\epsilon$ the error should be mainly due to the numeric
differentiation. But when the true derivative is small, the relative
error may be large (think of a true derivative which is exactly zero).
On the other hand, when a derivative is large in absolute value, the
absolute error may also be quite large.  Mind that the derivatives can
in practice be very small or very large, and also that a gradient
vector or a Hessian matrix often contain values that are not of the
same order of magnitude.

We check that either the \textit{absolute} error or the
\textit{relative} error is small. The idea is that none of these two
things can come by chance, and if one holds, the formula used must be
good even if the other criterion suggests an opposed conclusion. The
test is made \textit{elementwise} meaning that the relative error is
computed for each element of a gradient vector or Hessian matrix
ignoring the other elements.

\section{POT to NHPP parameterizations}

XXX To be completed

% Peak Over Threshold (POT) models are very popular in EVA. A threshold
% $u$. 

% \begin{itemize}
% \item Poisson-GP
% \item PP (for Point Process) or NHPP (for Non-Homogeneous Poisson
%   Proces)
% \end{itemize}

% The main difference is that 

% The \textbf{yaev} package provides the two transformations.

\section{EV distributions from other R packages}

The EV distributions are implemented in many R packages. A variety of
strategies regarding the problem $\xi = 0$ can be found. We now
describe these strategies and provide for each of them a ``code''
(\fbox{NT}, ...) that is used in Table~\ref{TabXi}, columns $\xi =
0$. Each strategy is briefly discussed.

\begin{enumerate}

\item \fbox{NT} Use only the formula for $\xi \neq 0$ i.e., {``do
    nothing''}.  In practice, an optimization or sampling algorithm
  will never come to the case $\xi = 0$ \textit{exactly} and this can
  only happen when the user gives this value e.g. as an initial value.
  There will be some numerical problems when $\xi$ is very small, say
  $\xi = \text{1e-14}$ or less. These problems are not so crucial for
  the usual probability functions: we get some wiggling when plotting
  the curves and zooming. Mind however that the random generation
  \code{rgev} or \code{rgpd} will produce silly results with very
  small $\xi$ if they use \code{qgev} or \code{qgpd}.
  
\item \fbox{$0.0$} Test the exact equality $\xi = 0$, and if this is true,
  \textit{switch to the exponential/Gumbel formula}.  This helps only
  when the user gives \code{xi = 0.0}, but we are essentially doing
  the same thing as in \fbox{NT}.

\item  \fbox{$\epsilon$/S}  Test the equality $ |\xi| \leqslant \epsilon$ where
  $\epsilon >0$ is very small, and if this is true, \textit{switch
    to the exponential/Gumbel formula}. So this produces a (very
  small) discontinuity.  E.g., \pkg{Renext} uses
  $\epsilon \approx \text{2e-14}$.
  
  
\item \fbox{$\epsilon$/AI} Test the equality
  $ |\xi| \leqslant \epsilon$ where $\epsilon >0$ is very small, and
  if this is true, use a dedicated \textit{approximation or
    interpolation}.  Several methods can be used including Taylor
  approximations. The discontinuity should then be undetectable.  Mind
  the probability functions although not being \textit{analytic}
  functions, are infinitely differentiable $C^\infty$ w.r.t. the
  parameters.
  
\end{enumerate}


\begin{table}
    \centering \small
    \begin{tabular}{| l || c | c | c | c | c || c | c | c | c | c |}
      \hline
      \multicolumn{1}{|c||}{\raisebox{-0.3em}{Package}}
      & \multicolumn{5}{c||}{GEV}
      & \multicolumn{5}{c|}{GPD}\\ \cline{2-6} \cline{7-11}
      \multicolumn{1}{|c||}{}
      & \multicolumn{1}{c|}{Lang.}
      & \multicolumn{1}{c|}{Vec. $\bs{\theta}$}
      & \multicolumn{1}{c|}{Grad.}
      & \multicolumn{1}{c|}{Hess.}
      & \multicolumn{1}{c||}{$\xi=0$}
      & \multicolumn{1}{c|}{Lang.}
      & \multicolumn{1}{c|}{Vec. $\bs{\theta}$}
      & \multicolumn{1}{c|}{Grad.}
      & \multicolumn{1}{c|}{Hess.}
      & \multicolumn{1}{c|}{$\xi=0$} \\ \hline\hline
      \textbf{evir}
      & R & no & no & no & NT & R & no & no & no & NT\\ \hline
      \textbf{evd}
      & R & no & no & no & 0.0 & R & no & no & no & 0.0\\ \hline
      \textbf{ismev}
      & R$^\star$ & no & no & no & NT & R$^\star$ & no & no & no & NT\\ \hline
      \textbf{Renext}
      &  &  &  &  &  & R & no & yes & yes & $\epsilon$/S \\ \hline
      \textbf{POT}
      &  &  &  &  &  & R$^\star$ & yes & no & no & 0.0 \\ \hline
      \textbf{SpatialExtremes}
      & R  & yes & no & no & $0.0$ & R & yes & no & no & $0.0$ \\ \hline   
      \textbf{revdbayes}
      & R  & yes & no & no & $\epsilon$/AI & R & yes & no & no & $\epsilon$/AI \\ \hline
      \textbf{mev}
      & R  & yes & yes$^\star$ & yes$^\star$& NT & R & yes & yes$^\star$
      & yes$^\star$ & NT \\ \hline
      \Gre{\textbf{yaev}}
      & \Gre{C} & \Gre{yes} & \Gre{yes} & \Gre{yes} & \Gre{$\epsilon$/AI}   
      & \Gre{C} & \Gre{yes} & \Gre{yes} & \Gre{yes} & \Gre{$\epsilon$/AI} \\
      \hline
    \end{tabular}
    \caption{\label{TabXi}\small \sf Features of some CRAN
      packages. \textit{Lang.}: the implementation language,
      \textit{Vec.}  $\bs{\theta}$: vectorized w.r.t. the
      parameters. The columns \textit{Grad.} and the \textit{Hess.}
      indicate if the gradient and Hessian are provided, and the
      columns $\epsilon = 0$ indicate the strategy used to cope with a
      zero or small shape, as described in the text. A star $\star$
      means that the functions are not exported.}
  \end{table}


\section*{Acknowledgments}

We are grateful to the authors and contributors of \textbf{Maxima} and
to the authors of the \textbf{maxiplot} \LaTeX{} package (J.M. Planas
and José Manuel Mira univ. de Murcia, Spain) which helped much for the
tedious computations required by the package.

\end{document}
